\documentclass{notes}

\class{MATH 131BH (Real Analysis)}

\notusingsubsection

\begin{document}
  \section{3.28 Monday Week 1: Intro to the course. Review of material covered in 131AH: foundations (definition and constructions of naturals and reals), metric space convergence, continuity.}
  
  \section{3.30 Wednesday Week 1: Limit of a function: definition and alternative formulations via images of balls and sequential characterization. Limit on a set, left and right limits for functions on \texorpdfstring{$\mathbb R$}{R}. Discontinuities of first and second kind. Monotone functions have no discontinuities of second kind.}
  
  \underline{{\boldmath \bfseries Limits of functions}}
  
  {\boldmath \bfseries Recall:} $f \colon X \to Y$ is said to be {\boldmath \bfseries continuous at $x \in X$} if 
  \[
    \forall \, \varepsilon > 0 \, \exists \, \delta > 0 \, \forall \, z \in x: \rho_X(x, z) < \delta \Rightarrow \rho(f(z), f(x)) < \varepsilon.
  \]

  Alternatives:
  \begin{itemize}
    \item $f(B_X(x, \delta)) \subseteq B_Y(f(x), \varepsilon)$; 
    \item $\forall \, \left \{ x_n \right \}_{n \in \mathbb N} \in X^\mathbb N: x_n \to x \Rightarrow f(x_n) \to f(x)$.
  \end{itemize}
  
  A function $f \colon X \to Y$ is {\boldmath \bfseries continuous} if 
  \[
    \forall \, x \in X: \text{$f$ is continuous at $x$, }
  \]
  or, alternatively,  
  \[
    \forall \, O \subseteq Y \text{ open}: f^{-1}(O) \text{ open.}
  \]

  \begin{defn}
    A function $f \colon X \to Y$ {\boldmath \bfseries has limit $y \in Y$ at $x \in X$}, notation $\lim_{z \to x} f(z) = y$, if 
    \[
      \forall \, \varepsilon > 0 \, \exists \, \delta > 0 \, \forall \, z \in X: 0 < \rho_X(x, z) < \delta \Rightarrow \rho_Y(f(z), y) < \varepsilon.
    \]
    
    Alternatives: 
    \begin{itemize}
      \item $f(B_X(x, \delta) \setminus \left \{ x \right \}) \subseteq B_Y(y, \varepsilon)$; 

      \item $\forall \, \left \{ x_n \right \}_{n \in \mathbb N} \in X^\mathbb N: (\forall \, n \in \mathbb N: x_n \neq x) \land x_n \to x \Rightarrow f(x_n) \to y$; 
        
      \item $g(z) := \begin{cases}
        f(z) & z \neq x \\ 
        y & z = x
      \end{cases}$ is continuous at $x$.
    \end{itemize}
  \end{defn}
  
  \newpage

  \begin{defn}
    $f$ has a {\boldmath \bfseries removable discontinuity} at $x$ if $\lim_{z \to x} f(z)$ exists but $\neq f(x)$.
  \end{defn}
  
  % todo: is this a defn?
  \begin{defn}
    Let $A \subseteq X$ be nonempty, $x \in \overline A$ be not an isolated point.
    Then $\lim_{z \to x} f(z) = \lim_{z \to x} f_A(z)$ where $f_A$ is the restriction of $f$ to $A$.
  \end{defn}
  
  \begin{defn}
    For $f \colon \mathbb R \to \mathbb R$, let $x \in \overline{\operatorname{Dom}(f)}$ be such that $\operatorname{Dom}(f) \cap (x, \infty) \neq \varnothing$ and $\operatorname{Dom}(f) \cap (-\infty, x) \neq \varnothing$.
    Then $\lim_{z \to x^+} f(z) := \lim_{z \to x, z \in \operatorname{Dom}(f) \cap (x, \infty)} f(z)$ and $\lim_{z \to x^-} f(z) := \lim_{z \to x, z \in \operatorname{Dom}(f) \cap (-\infty, x)} f(z)$ are the {\boldmath \bfseries right / left limits of $f$ at $x$}.

    Alternative notation: $f(x^+)$, $f(x^-)$.
  \end{defn}
  
  \begin{eg}
    \[
      f(x) = \begin{cases}
        1 & x \in \mathbb Q \\ 
        0 & x \not \in \mathbb Q
      \end{cases}
    \]
    has no right or left limits.
  \end{eg}
  
  \begin{eg}
    \[
      f(x) = \begin{cases}
        \frac{1}{n + 1} & x = q_n \text{ where $\left \{ q_n \right \}_{n \in \mathbb N}$ enumerates $\mathbb Q$} \\ 
        0 & x \not \in \mathbb Q.
      \end{cases}
    \]
    Then $\forall \, x \not \in \mathbb Q: \lim_{z \to x} f(z) = 0$ so $f$ is continuous on $\mathbb R \setminus \mathbb Q$, and $\forall \, x \in \mathbb Q: \lim_{z \to x} f(z) = 0$ but $f$ is not continuous at $x$.
  \end{eg}
  
  \begin{lem}
    \[
      \forall \, r > 0 \, \forall \, \varepsilon > 0: \left \{ x \in \mathbb R : \left | x \right | < r \land \left | f(x) \right | > \varepsilon \right \} \text{finite} \Rightarrow \forall \, x \in \mathbb R: \lim_{z \to x} f(z) = 0.
    \]
  \end{lem}
  
  \begin{defn}
    A function $f \colon \mathbb R \to \mathbb R$ has a {\boldmath \bfseries discontinuity of }
    \begin{itemize}
      \item {\boldmath \bfseries first kind} at $x$ if $f(x^+)$ and $f(x^-)$ exist but are not both equal to $f(x)$;

      \item {\boldmath \bfseries second kind} at $x$ if one or both of $f(x^+)$ and $f(x^-)$ don't exist.
    \end{itemize}
  \end{defn}
  
  \newpage
  
  \begin{eg}
    \[
      f(x) := \begin{cases}
        (-1)^n & x = \frac{1}{n + 1}, n \in \mathbb N \\ 
        \text{linear} & x \in (0, \infty) \setminus \left \{ \frac{1}{n + 1} : n \in \mathbb N \right \} \\ 
        0 & x \leq 0.
      \end{cases}
    \]
    
    % todo: add figure?
    
    This function has a discontinuity of second kind at 0.
  \end{eg}

  \begin{lem}
    Let $f \colon \mathbb R \to \mathbb R$ ($\operatorname{Dom}(f) = \mathbb R$) be monotone.
    Then $\forall \, x \in \mathbb R: \text{$f(x^+), f(x^-)$ exist}$ and so $f$ has no discontinuities of second kind.
  \end{lem}
  
  \begin{prf}
    Let $x \in \mathbb R$ and assume $f$ is nondecreasing.
    We claim that $\lim_{z \to x^+} f(z) = \inf \left \{ f(z) : z > x \right \}=: L$.

    Indeed, $\forall \, z > x: f(z) \geq f(x)$, so $L \geq f(x)$ and so $L \in \mathbb R$.
    Then $(\forall \, z > x: L \leq f(z)) \land (\forall \, \varepsilon > 0 \, \exists \, z_\varepsilon  > x: f(z_\varepsilon) < L + \varepsilon)$.
    Let $\delta := z_\varepsilon - x$.
    Then $\forall \, z \in (x, x + \delta): f(z) \leq f(z_\varepsilon) < L + \varepsilon$.
    Then $\forall \, z \in (x, x + \delta): L \leq f(z) < L + \varepsilon$ and therefore $\left | f(z) - L \right | < \varepsilon$.
    Then $\lim_{z \to x^+} f(z) = L$.
  \end{prf}
  
  \section{3.31 Thursday Week 1: Monotone functions have only countably many discontinuities. Functions of bounded variation. Jordan decomposition theorem. Comments on uniqueness. Rectifiability of curves. Limsup and liminf of a function.}
  
  \underline{\boldmath \bfseries Limits of functions}
  
  Last time we showed that monotone functions have no discontinuities of second kind.

  \begin{lem}
    Let $f \colon \mathbb R \to \mathbb R$ be monotone.
    Then $\left \{ x \in \mathbb R : f(x^+) \neq f(x^-) \right \}$ is countable.
  \end{lem}
  
  \begin{prf}
    Pick $k, m \in \mathbb N$ and let $A_{m, k} := \left \{ x \in [-m, m] : \left | f(x^+) - f(x^-) \right | > \frac{1}{k + 1} \right \}$.
    We claim that $A_{m, k}$ is finite.
    
    Let $x_{0} < x_{1} < \cdots < x_n$ be such that $\forall \, i \leq n: x_i \in A_{k, m}$.
    Assume (without loss of generality) that $f$ is non-decreasing.
    Then 
    \begin{align}
      f(m + 1) &\geq f(x_n^+) = f(x_0^+) + \sum_{i = 1}^n \left ( f(x_i^+) - f(x_{i - 1}^+) \right ) \nonumber \\ 
      &\geq f(m - 1) + \sum_{i = 1}^n \left ( f(x_i^+) - f(x_i^-) \right ) \nonumber \\ 
      &\geq f(-m + 1) + \frac{n}{k + 1}.
    \end{align}
    Then $n \leq (k + 1)$.
    Since $\left \{ x \in \mathbb R : f(x^+) \neq f(x^-) \right \} = \bigcup_{k \in \mathbb N} \bigcup_{m \in \mathbb N} A_{k, m}$, we are done.
  \end{prf}
  
  {\boldmath \bfseries  Q:} Can these be generalized to other functions?

  \begin{defn}
    A {\boldmath \bfseries partition} $\Pi$ of an interval $[a, b]$ is a sequence $\left \{ t_i \right \}_{i = 0}^n$ such that 
    \[
      a = t_0 < t_{1} < \cdots < t_{n - 1} < t_n = b.
    \]
  \end{defn}
  
  \begin{defn}
    Given $f \colon [a, b] \to \mathbb R$, its {\boldmath \bfseries total variation} on $[a, b]$
    \[
      V(f, [a, b]) := \sup_{\Pi = \left \{ t_i \right \}_{i = 0}^n} \sum_{i = 1}^n \left | f(t_i) - f(t_{i - 1}) \right | 
    \]
    where the supremum is over the partitions of $[a, b]$.
  \end{defn}
  
  \begin{defn}
    $f$ is said to be of {\boldmath \bfseries bounded variation} on $[a, b]$ if $V(f, [a, b]) < \infty$.
  \end{defn}
  
  \begin{lem}
    If $f \colon \mathbb R \to \mathbb R$ is of bounded variation on $[-m, m]$ for all $m \in \mathbb N$, then $f$ has only discontinuities of first kind and the set $\left \{ x \in \mathbb R : f(x^+) \neq f(x^-) \right \}$ is countable.
  \end{lem}
  
  \begin{thm}[Jordan decomposition (1881)]
    Let $f \colon [a, b] \to \mathbb R$ obey $V(f, [a, b]) < \infty$.
    Then $\exists \, h, g \colon [a, b] \to \mathbb R$ nondecreasing such that $\forall \, t \in [a, b]: f(t) = h(t) - g(t)$.
  \end{thm}
  
  \begin{prf}
    Define $h(t) := V(f, [a, t])$ and $g(t) := V(f, [a, t]) - f(t)$.
    Note that $h(t) - g(t) = f(t)$.

    We need to show that $h$ and $g$ are nondecreasing.
    
    Let $a \leq t < t' \leq b$.
    Then for any partition $\Pi$ of $[a, t]$, $\Pi' = \Pi \cup \left \{ t' \right \}$ is a partition of $[a, t']$.
    Then 
    \[
      V(f, [a, t']) \geq \sum_{i = 1}^m \left | f(t_i) - f(t_{i - 1}) \right | + \left | f(t') - f(t) \right |.
    \]
    Taking supremum over $\Pi$ gives 
    \[
      V(f, [a, t']) \geq V(f, [a, t]) + \left | f(t') - f(t) \right |.
    \]
    Note that $\left | f(t') - f(t) \right | \geq 0$ and $\left | f(t') - f(t) \right | \geq f(t') - f(t)$.
    Then $h(t') \geq h(t)$ and $g(t') \geq g(t)$.
  \end{prf}
  
  \strut
  
  The representation of $f = h - g$ is called a Jordan decomposition.
  This is not unique because a nondecreasing function can be added to both $h$ and $g$.
  
  However, there is a minimal decomposition $f = h_{0} - g_{0}$ such that $g_0(a) = 0$ such that for any other Jordan decomposition $f = h - g$ we have $h - h_0$, $g - g_0$ nondecreasing.
  This is then \textit{the} Jordan decomposition.
  
  \strut
  
  \underline{{\boldmath \bfseries Rectifiability of curves}}
  
  \begin{defn}
    Let $(X, \rho)$ be a metric space.
    A curve $\mathcal C$ is $\operatorname{Ran}(f)$ for an $f \colon \mathbb R \to X$ continuous such that $\operatorname{Dom}(f)$ is nonempty and connected.
    This $f$ is called a {\boldmath \bfseries parametrization} of $\mathcal C$.
  \end{defn}
  
  \begin{defn}
    Assuming $\operatorname{Dom}(f) = [a, b]$, the {\boldmath \bfseries length of $\mathcal C$} is 
    \[
      \ell(\mathcal C) := \sup_{\Pi = \left \{ t_i \right \}_{i = 0}^n} \sum_{i = 1}^n \rho(f(t_{i - 1}), f(t_i)).
    \]
  \end{defn}
  
  \begin{defn}
    A curve is {\boldmath \bfseries rectifiable} if $\ell(\mathcal C) < \infty$.
  \end{defn}
  
  \begin{defn}
    Let $(X, \rho)$ be a metric space and $f \colon X \to \mathbb R$.
    Then 
    \[
      \limsup_{z \to x} f(z) := \inf_{\delta > 0} \sup_{z \in B(x, \delta) \setminus \left \{ x \right \}} f(z)
    \]
    and 
    \[
      \liminf_{z \to x} f(z) := \sup_{\delta > 0} \inf_{z \in B(x, \delta) \setminus \left \{ x \right \}} f(z).
    \]
  \end{defn}
  
  \begin{lem}
    \[
      \lim_{z \to x} f(z) \text{ exists in $\mathbb R$} \Leftrightarrow \limsup_{z \to x} f(z) = \liminf_{z \to x} f(z) \in \mathbb R.
    \]
  \end{lem}
  
  \newpage
  
  \section{4.1 Friday Week 1: Discussion}
  
  \begin{defn}
    Let $(X, \rho_X)$, $(Y, \rho_Y)$ be metric spaces, $E \subseteq X$, $f \colon E \to Y$, and $x \in \overline E$.
    Then $\lim_{t \to x} f(t) = \alpha$ is defined by 
    \[
      \forall \, \varepsilon > 0 \, \exists \, \delta > 0: \forall \, x \in E \land 0 < \rho_X(t, x) < \delta \Rightarrow \rho_Y(f(t), \alpha) < \varepsilon.
    \]
    
    Equivalently, 
    \[ 
      \forall \, \left \{ t_n \right \}_{n \in \mathbb N} \in (E \setminus \left \{ x \right \})^\mathbb N: t_n \to x \Rightarrow f(t_n) \to \alpha.
    \]

    \begin{note}
      $f$ need not be defined at $x$.
    \end{note}
  \end{defn}
  
  \begin{rmk}
    \[
      \limsup_{t \to x} f(t) := \inf_{\delta > 0} \sup_{t \in B(x, \delta) \setminus \left \{ x \right \}} f(t) = \lim_{\delta \to 0} \sup_{t \in B(x, \delta) \setminus \left \{ x \right \}} f(t).
    \]
    $\liminf$ is similarly defined.
  \end{rmk}
  
  \begin{rmk}
    \[
      \limsup = \liminf \Rightarrow \text{$\lim$ exists.}
    \]
  \end{rmk}
  
  \underline{{\boldmath \bfseries Discontiuities}}
  
  \begin{defn}
    Let $f \colon (a, b) \to \mathbb R$ be not continuous at $x$. Then $f$ has a {\boldmath \bfseries discontinuity of first kind} at $x$ if $f(x+)$ and $f(x-)$ both exist.
    Otherwise it is of {\boldmath \bfseries second kind}.
  \end{defn}
  
  \begin{rmk}
    Discontinuities of first kind are also known as {\boldmath \bfseries simple discontinuities}.
    The cases include 
    \begin{itemize}
      \item $f(x+) = f(x-) \neq f(x)$: {\boldmath \bfseries removable discontinuity}, and  

      \item $f(x+) \neq f(x-)$: {\boldmath \bfseries jump discontinuity}.
    \end{itemize}
  \end{rmk}
  
  \begin{eg}
    \[
      f(x) = \begin{cases}
        \sin \left ( \frac{1}{x} \right ) & x \neq 0 \\ 
        0 & x = 0
      \end{cases}
    \]
    has a discontinuity of second kind at 0.
  \end{eg}
  
  \begin{eg}
    \[
      f(x) = \begin{cases}
        \frac{1}{q} & x = \frac{p}{q} \in \mathbb Q \\ 
        0 & x \in \mathbb R \setminus \mathbb Q
      \end{cases}
    \]
    is continuous on $\mathbb R \setminus \mathbb Q$ and has discontinuities of first kind (removable) at every point in $\mathbb Q$.
  \end{eg}
  
  {\boldmath \bfseries Recall:} A monotone function has no discontinuity of second kind and has at most countably many discontinuities of first kind.
  One can deduce this from the fact that the real line is a union of countably many open intervals (indexed by rationals).
  
  \begin{defn}
    A function $f \colon (a, b) \to \mathbb R$ is convex if 
    \[
      \forall \, x, y \in (a, b): x \leq y \Rightarrow (\forall \, \lambda \in [0, 1]: f(\lambda x + (1 - \lambda) y)) \leq \lambda f(x) + (1 - \lambda) f(y).
    \]
    In words, this means that for any interval, the secant line is above the graph.
  \end{defn}
  
  \section{4.4 Monday Week 2: Existence of limit is equivalent to equality and finiteness of limsup and liminf. Derivative of a real valued function of one real variable. Differentiability implies continuity. Connection with linear approximation. Sum and product rule, chain rule and inverse function rule. First-derivative test and discussion of important counterexamples.}
  
  {\boldmath \bfseries \underline{Last time}:} $\lim_{z \to x} f(z)$, $\limsup_{z \to x} f(z) = \inf_{\delta > 0} \sup_{z \in B(x, \delta) \setminus \left \{ x \right \}x} f(z)$
  
  \begin{lem}
    \[
      \lim_{z \to x} f(z) \text{ exists (in $\mathbb R$)} \Leftrightarrow \limsup_{z \to x} f(z) = \liminf_{z \to x} f(z) \in \mathbb R.
    \]
  \end{lem}
  
  \begin{prf}
    Both are equivalent: 
    \[
      \forall \, \varepsilon > 0 \, \exists \, \delta > 0: 0 \leq \sup_{z \in B(x, \delta) \setminus \left \{ x \right \}} f(z) - \inf_{z \in B(x, \delta) \setminus \left \{ x \right \}} f(z) \leq 2 \varepsilon.
    \]
  \end{prf}
  
  \begin{defn}
    \[
      \lim_{z \to x} f(z) = \begin{cases}
        +\infty & \limsup_{z \to x} f(z) = \liminf_{z \to x} f(z) = +\infty \\
        -\infty & \limsup_{z \to x} f(z) = \liminf_{z \to x} f(z) = -\infty.
      \end{cases}
    \]
  \end{defn}
  
  \begin{note}
    This characterization works even outside $\mathbb R$-valued functions:
    \[
      \lim_{z \to x} f(z) \text{ exists} \Leftrightarrow \lim_{\delta \to 0^+} \sup \underbrace{\left \{ \rho(f(z), f(u)) : z, u \in B(x, \delta) \setminus \left \{ x \right \} \right \}}_{= \operatorname{diam}(f(B(x, \delta) \setminus \left \{ x \right \}))} = 0.
    \]
  \end{note}
  
  \newpage
  
  \underline{\boldmath \bfseries The derivative}

  \begin{defn}
    Let $f \colon \mathbb R \to \mathbb R$, $x \in \operatorname{int} (\operatorname{Dom}(f))$.
    We say that $f$ has {\boldmath \bfseries derivative} or {\boldmath \bfseries is differentiable at $x$} if 
    \[
      f'(x) := \lim_{z \to x} \frac{f(z) - f(x)}{z - x} \text{ exists in $\mathbb R$}.
    \]
    
    We call $f'(x)$ (Lagrange notation) the {\boldmath \bfseries derivative at $x$}, alternative notation $\frac{df}{dx}$ (Leibniz notation).
  \end{defn}
  
  \begin{lem}
    \[
      f'(x) \text{ exists} \Rightarrow \text{$f$ continuous at $x$}.
    \]
  \end{lem}
  
  \begin{prf}
    The existence of $f'(x)$ implies that $\exists \, \delta_0 > 0 \, \forall \, z \in \mathbb R: 0 < \left | z - x \right | < \delta_0 \Rightarrow \left | \frac{f(z) - f(x)}{z - x} \right | \leq 1 + \left | f'(x) \right |$.
    Then, choosing $\varepsilon > 0$ and letting $\delta := \frac{\varepsilon}{1 + \left | f'(x) \right |}$, we get 
    \[
      \forall \, z \in \mathbb R: 0 < \left | z - x \right | < \delta \mathbb \Rightarrow \left | f(z) - f(x) \right | \leq (1 + \left | f'(x) \right |) \left | z - x \right | < (1 + \left | f'(x) \right |) \frac{\epsilon}{1 + \left | f'(x) \right |} = \varepsilon.
    \]
    Since $f(z) - f(x) = 0$ for $z = x$, we are done (in fact, we have shown that $f$ is lipschitz continuous).
  \end{prf}
  
  Another way to write existence of $f'(x)$: 
  \[
    f(z) - f(x) = (f'(x) + u_x(z)) (z - x)
  \]
  where $\lim_{z \to x} u_x(z) = 0$.
  (Just define: $u_x(z) := \frac{f(z) - f(x)}{z - x} - f'(x)$ for $z \neq x$)
  
  \begin{lem}[Linear approximation]
    \[
      \text{$f'(x)$ exists} \Leftrightarrow \exists \, L \in \mathbb R: \lim_{\delta \to 0^+} \sup_{\left | z - x \right | < \delta} \frac{1}{\delta} \left | f(z) - f(x) - L (z - x) \right | = 0.
    \]
  \end{lem}
  
  \begin{lem}[Sum \& product rule]
    Let $f, g$ be differentiable at $x$.
    Then so are $f + g$ and $f \cdot g$ and 
    \begin{align*}
      (f + g)'(x) &= f'(x) + g'(x) \\ 
      (f \cdot g)'(x) &= f'(x) g(x) + g'(x) f(x) \quad \text{(Leibniz rule)}.
    \end{align*}
  \end{lem}
  
  \begin{prf} For product rule, note that  
    \[
      f(z) g(z) - f(x) g(x) = (f(z) - f(x)) g(z) + (g(z) - g(x)) f(z).
    \]
    Then 
    \[
      \frac{f(z) g(z) - f(x) g(x)}{z - x} = \frac{f(z) - f(x)}{z - x} g(z) + \frac{g(z) - g(x)}{z - x} f(z).
    \]
    Since $g(z) \to g(x)$ by continuity of $g$, formula follows by sum \& product rule for limit.
  \end{prf}
  
  \begin{lem}[Chain rule]
    Let $f$ be differentiable at $x$ and $g$ at $f(x)$.
    Then $g \circ f$ is differentiable at $x$ and 
    \[
      (g \circ f)'(x) = g'(f(x)) f'(x) \quad \left ( \frac{dg}{df} \frac{df}{dx} \right )
    \]
  \end{lem}
  
  \begin{prf}
    Define $v_{f(x)}$ such that $g(y) - g(f(x)) = (g'(f(x))) + v_{f(x)}(y)) (y  - f(x))$ and $u_x$ such that $f(z) - f(x) = (f'(x) + u_x(z)) (z - x)$.
    \begin{align*}
      (g \circ f)(z) - (g \circ f)(x) &= [g'(f(x)) + v_{f(x)}(f(z))] (f(z) - f(x)) \\ 
      &= [g'(f(x)) + v_{f(x)}(f(z))] [f'(x) + u_x(z)] (z - x)
    \end{align*}
    Dividing by $z - x \neq 0$, note that $f(z) \to f(x)$ implies $v_{f(x)}(f(z)) \to 0$ as $z \to x$, we are done.
  \end{prf}
  
  \begin{lem}
    Let $f \colon \mathbb R \to \mathbb R$ be injective on $\operatorname{Dom}(f)$ and differentiable at $x \in \operatorname{int}(\operatorname{Dom}(f))$.
    Assume $f'(x) \neq 0$ and $f(x) \in \operatorname{int}(\operatorname{Ran}(f))$.
    Then $f^{-1}$ is differentiable at $f(x)$ and 
    \[
      (f^{-1})'(f(x)) = \frac{1}{f'(x)}.
    \]
    
    In Leibniz notation: 
    \[
      \frac{dy}{dx} = \frac{1}{\frac{dx}{dy}}.
    \]
  \end{lem}
  
  \begin{lem}[First derivative test]
    Let $f \colon [a, b] \to \mathbb R$ be continuous on $[a, b]$ and differentiable on $(a, b)$.
    Then if $x \in (a, b)$ is a local maximum of $f$ (i.e. $\exists \, \delta > 0 \, \forall \, z \in \mathbb R: \left | z - x \right | < \delta \Rightarrow f(x) \geq f(z)$) then $f'(x) = 0$.
  \end{lem}
  
  \begin{prf}
    \[
      z > x \land \left | z - x \right | < \delta \Rightarrow \frac{f(z) - f(x)}{z - x} \leq 0 \Rightarrow f'(x) \leq 0
    \]
    and 
    \[
      z < x \land \left | z - x \right | < \delta \Rightarrow \frac{f(z) - f(x)}{z - x} \geq 0 \Rightarrow f'(x) \geq 0.
    \]
  \end{prf}
  
  \section{4.6 Wednesday Week 2: Discussion}
  
  {\boldmath \bfseries Recall:} For, $x \colon [a, b] \to \mathbb R$, the total variation
  \[
    V(f, [a, b]) = \sup_{\Pi} \sum_{i = 1}^n \left | f(x_i) -f(x_{i - 1})  \right |
  \]
  where $\Pi = \left \{ a = x_0 < x_1 < \cdots < x_n = b  \right \}$.
  We say $f \in BV([a, b])$ if $V(f, [a, b]) < \infty$.
  
  \begin{thm}[Jordan decomposition]
    \[
      \forall \, f \in BV([a, b]) \, \exists \, h, g \colon [a, b] \to \mathbb R \text{ nondecreasing}: f = h - g.
    \]
  \end{thm}
  
  \begin{cor}
    $f \in BV([a, b])$ can only have discontinuities of first kind and countably many of them.
  \end{cor}
  
  \begin{eg}
    $f(x) = \sin x \in BV([-1, 1])$ since $f$ is nondecreasing on $[-1, 1]$ and hence $V(f, [a, b]) = f(b) - f(a)$.
  \end{eg}
  
  \begin{eg}
    $f(x) = \sin x \in BV([-M, M])$ by additive property of $V$.
  \end{eg}
  
  {\boldmath \bfseries Q.} Does $BV([a, b])$ imply bounded on $[a, b]$?

  Yes.
  By triangle inequality, 
  \[
    \left | f(x) \right | \leq \left | f(a) \right | + \left | f(a) - f(x) \right | \leq \left | f(a) \right | + V(f, [a, b]) < \infty.
  \]
  
  {\boldmath \bfseries Q.} Does being bounded on $[a, b]$ imply $BV([a, b])$.

  No.
  A counterexample is 
  \[
    f(x) = \begin{cases}
      \sin \frac{1}{x} & x \neq 0 \\ 
      0 & x = 0
    \end{cases}
  \]
  on $[0, 1]$.
  
  Choose $x_n = 1 / (n \pi / 2)$ such that $\sin(1 / x_n) = \sin(n \pi / 2)$.
  Then $\sum_{i = 1}^{2 n} \left | f(x_i) - f(x_{i - 1}) \right | = \sum_{k = 1}^n \left | f(x_{2 k + 1}) \right | = n \to \infty$.
  
  \begin{eg}
    Is 
    \[
      f(x) = \begin{cases}
        x \sin \frac{1}{x} & x \neq 0 \\ 
        0 & x = 0
      \end{cases}
    \]
    on $[0, 1]$ of bounded variation?
    
    No.
    Choose the same $x_n$ as above.
    Note that $f(x_n) = \frac{2}{n \pi} \sin(n \pi / 2)$.
    Then $\sum_{i = 1}^{2 n} \left | f(x_i) - f(x_{i - 1}) \right | = \sum_{k = 1}^n \frac{2}{(2 k - 1) \pi} \to \infty$.
  \end{eg}
  
  \begin{eg}
    Is 
    \[
      f(x) = \begin{cases}
        x^2 \sin \frac{1}{x} & x \neq 0 \\ 
        0 & x = 0
      \end{cases}
    \]
    on $[0, 1]$ of bounded variation?
    
    Yes.
    Note that 
    \[
      f'(0) = \lim_{t \to 0} \frac{t^2 \sin \frac{1}{t} - 0}{t} = \lim_{t \to 0} t \sin \frac{1}{t} = 0.
    \]
    
    Note that for $x \neq 0$, 
    \[
      f'(x) = 2 x \sin \frac{1}{x} + x^2 \left ( -\frac{1}{x^2} \right ) \cos \frac{1}{x} = 2 x \sin \frac{1}{x} - \cos \frac{1}{x}
    \]
    is bounded: $\left | f'(x) \right | \leq 2 \left | x \right | + 1 \leq 3$.

    Note that by mean value theorem, we have 
    \[
      \sum \left | f(x_i) - f(x_{i - 1}) \right | \leq \sum \left | f'(\xi) \right | (x_i - x_{i - 1}) \leq M (b - a) < \infty
    \]
    where $\left | f'(\xi) \right | \leq M$.

    Then $f$ is of bounded variation on $[0, 1]$.
  \end{eg}
  
  \begin{thm}
    If $f'$ exists and is bounded on $[a, b]$ then $f$ is of bounded variation.
  \end{thm}
  
  {\boldmath \bfseries Q.} Does the existence $f'$ on $[a, b]$ and $f$ being of bounded variation on $[a, b]$ imply $f'$ is bounded on $[a, b]$?
  
  \section{4.7 Thursay Week 2: Mean-Value Theorems of Rolle, Lagrange and Cauchy. Applications: Monotone differentiable functions have \texorpdfstring{\newline}{}derivative of one sign. Derivative of a differentiable function has no discontinuities of first kind (but those of second kind can occur densely). L'Hospital's Rule and its proof from Cauchy's MVT.}

  {\boldmath \bfseries \underline{Mean value theorems}}

  {\boldmath \bfseries Last time:} $f'(x) = $ derivative is linked to the local maxima and minima (first derivative test).

  \begin{thm}[Mean value theorem]
    Let $f \colon [a, b] \to \mathbb R$ be continuous on $[a, b]$ and differentiable on $(a, b)$.
    Then 
    \begin{enumerate}
      \item (Rolle's theorem, 1691) $f(a) = f(b) \Rightarrow \exists \, x \in (a, b): f'(x) = 0$, 

      \item (Lagrange's mean value theorem) $\exists \, x \in (a, b): f'(x) = \frac{f(b) - f(a)}{b - a}$, and 

      \item (Cauchy mean value theorem, 1823) if also $g \colon [a, b] \to \mathbb R$ is continuous on $[a, b]$ and differentiable on $(a, b)$, then 
      \[
        \forall \, x \in (a, b): g'(x) \neq 0 \Rightarrow g(a) \neq g(b) \land \exists \, x \in (a, b): \frac{f'(x)}{g'(x)} = \frac{f(b) - f(a)}{g(b) - g(a)}.
      \]
    \end{enumerate}
  \end{thm}
  
  \begin{prf}
    \begin{enumerate}
      \item $f(a) = f(b)$ $\land$ continuous function on $[a, b]$ achieves one of maximum and minimum on $(a, b)$ $\Rightarrow \exists \, x \in (a, b): \text{$x$ is local maximum or local minimum of $f$}$.
      Then $f'(x) = 0$.
      
      \item Let $h(x) = f(x) - \frac{f(b) - f(a)}{b - a} (x - a)$.
      Then $h(a) = f(a)$, $h(b) = f(b) - \frac{f(b) - f(a)}{b - a} (b - a) = f(a)$.
      Then, by 1., $\exists \, x \in (a, b): h'(x) = f'(x) - \frac{f(b) - f(a)}{b - a} = 0$.
      
      \item Let $h(x) = f(x) - \frac{f(b) - f(a)}{g(b) - g(a)} (g(x) - g(a))$.
      Note that this is well defined since by 1. we have $g(b) \neq g(x)$. % todo: g(a)??
      Then $h(a) = f(a) = h(b)$ so by 1. we have $\exists \, x \in (a, b): h'(x) = f'(x) - \frac{f(b) - f(a)}{g(b) - g(a)} g'(x) = 0$.
    \end{enumerate}
  \end{prf}
  
  {\boldmath \bfseries \underline{Applications}}
  
  \begin{lem}
    Let $f \colon [a, b] \to \mathbb R$ be continuous on $[a, b]$ and differentiable on $(a, b)$.
    Then 
    \[
      \forall \, x \in (a, b): f'(x) \geq 0 \Leftrightarrow \forall \, x, y \in [a, b]: x \leq y \Rightarrow f(x) \leq f(y).
    \]
  \end{lem}
  
  \begin{prf}
    The $\Leftarrow$ direction is immediate from the definition of limit $\left ( \frac{f(y) - f(x)}{y - x} \geq 0 \right )$.
    
    For the $\Rightarrow$ direction, if $\exists \, x \geq y: f(y) < f(x)$ then by the mean value theorem $\exists \, z \in (x, y): f'(z) = \frac{f(y) - f(x)}{y - x} < 0$.
  \end{prf}
  
  \section{4.8 Friday Week 2: Taylor's theorem via Mean Value Theorem (Rolle suffices). Riemann integral: motivation, definitions of marked partition, mesh of partition and Riemann sum. Notion of a function being Riemann integrable. Linearity of integral.}

  {\boldmath \bfseries \underline{Taylor's theorem}}
  
  \begin{defn}[Higher order derivatites]
    Define $f^{(0)} := f$ and for all $n \in \mathbb N$ define $f^{(n + 1)}(x) := \left ( f^{(n)} \right )'(x)$ assuming the derivatives exist.
    We call $f^{(n)}$ the {\boldmath \bfseries $n$-th derivative of $f$}.
  \end{defn}
  
  \begin{thm}[Taylor's theorem (Taylor 1715, Gregory 1671)]
    Let $n \in \mathbb N$ and $f \colon (a, b) \to \mathbb R$ an $(n + 1)$-times differentiable function.
    Then 
    \[
      \forall \, x_0 \in (a, b) \, \forall \, x \in (x_0, b) \, \exists \, \xi \in (x_0, x): f(x) = \underbrace{\sum_{k = 0}^n \frac{f^{(k)}(x_0)}{k!} (x - x_0)^k}_{\text{$n$-th order Taylor polynomial at $x_0$}} {} + \frac{f^{(n + 1)}(\xi)}{(n + 1)!} (x - x_0)^{n + 1}.
    \]
  \end{thm}
  
  \begin{prf}
    Based on MVT.
    
    Denote
    \[
      P_n(z) := \sum_{k = 0}^n \frac{f^{(k)}(x_0)}{k!} (z - x_0)^k.
    \]
    

    Pick $x \in (x_0, b)$ and denote 
    \[
      A := \frac{f(x) - P_n(x)}{(x - x_0)^{n + 1}}.
    \]
    
    Set 
    \[
      h(z) := f(z) - P_n(z) - A(z - x_0)^{n + 1}.
    \]
    
    Note that 
    \[
      \forall \, k \in \mathbb N: k \leq n \Rightarrow f^{(k)}(x_0) = 0.
    \]
    
    We claim that 
    \[
      \forall \, k \in \mathbb N: 1 \leq k \leq n + 1 \Rightarrow \exists \, \xi_k \in (x_0, x): h^{(k)}(\xi_k) = 0.
    \]
    
    For $k = 1$, the choice of $A$ implies $h(x) = 0$ so since $h(x_0) = 0$, by Rolle's theorem
    \[
      \exists \, \xi_1 \in (x_0, x): h'(\xi) = 0.
    \]
    
    Assume true for some $k \in \mathbb N$ such that $1 \leq k \leq n$.
    Then $h^{(k)}(x_0) = 0$ and $h^{(k)}(\xi_k) = 0$ for $\xi_k \in (x_0, x)$.
    Then by Rolle's theorem 
    \[
      \exists \, \xi_{k + 1} \in (x_0, \xi_k): h^{(n + 1)}(\xi_{k + 1}) = 0.
    \]
    
    Now observe that $P_n^{(n + 1)} = 0$.
    Then $0 = h^{(n + 1)}(\xi_{n + 1}) = f^{(n + 1)}(\xi_{n + 1}) - A (n + 1)!$.
    Then 
    \[
      f(x) - P_n(x) = A(x - x_0)^{n + 1} = \frac{f^{(n + 1)}(\xi_{n + 1})}{(n + 1)!} (x - x_0)^{n + 1}.
    \]
  \end{prf}
  
  {\boldmath \bfseries \underline{Riemann integral}} (Riemann 1854)
  
  {\boldmath \bfseries Goal:} Given $f \colon [a, b] \to \mathbb R$, assign meaning to the area under the graph of $f$ on $[a, b]$; namely to the set 
  \[
    \left \{ (x, y) \in \mathbb R^2 : x \in [a, b] \land 0 \leq y \leq f(x) \right \} \quad \text{(for $f \geq 0$)}.
  \]
  
  {\boldmath \bfseries Idea:} Approximate $f$ with a piecewise constant function and use that the area of a rectangle is ``known.''
  
  \begin{defn}
    Given $[a, b]$, a {\boldmath \bfseries marked partition $\Pi$} of $[a, b]$ is two sequences $\left \{ t_i \right \}_{i = 0}^n$, $\left \{ t^*_i \right \}_{i = 1}^n$ such that 
    \begin{itemize}
      \item $a = t_0 < t_1 < \dots < t_{n - 1} < t_n = b$ and 

      \item $\forall \, i = 1, \dots, n: t^*_i \in [t_{i - 1}, t_i]$.
    \end{itemize}
  \end{defn}
  
  \begin{defn}
    The {\boldmath \bfseries mesh of $\Pi$} is defined by $|| \Pi || := \max_{i = 1, \dots, n} \left | t_i - t_{i - 1} \right |$.
  \end{defn}
  
  \begin{defn}
    Given $f \colon [a, b] \to \mathbb R$ and a marked partition $\Pi$, the associated {\boldmath \bfseries Riemann sum} is 
    \[
      R(f, \Pi) := \sum_{i = 1}^n f(t^*_i) (t_i - t_{i - 1}).
    \]
  \end{defn}
  
  \newpage
  
  \begin{defn}
    A function $f \colon [a, b] \to \mathbb R$ is said to be {\boldmath \bfseries Riemann integrable} (on $[a, b]$) if there exists $L \in \mathbb R$ such that 
    \[
      \forall \, \varepsilon > 0 \, \exists \, \delta > 0 \, \forall \, \Pi = \text{marked partition of $[a, b]$}: || \Pi || < \delta \Rightarrow \left | R(f, \Pi) - L \right | < \varepsilon.
    \]
    
    We sometimes write this as $\lim_{|| \Pi || \to 0} R(f, \Pi) = L$ (this $L$ is unique).
    Notation for $L$ is $\int_a^b f(x)\ dx$.
  \end{defn}
  
  \begin{lem}[Additivity and homogeneity of Reimann integral]
    Let $f, g \colon [a, b] \to \mathbb R$ be Riemann integrable on $[a, b]$.
    Let $\alpha, \beta \in \mathbb R$.
    Then $\alpha f + \beta g$ is Riemann integrable on $[a, b]$ and 
    \[
      \int_a^b (\alpha f(x) + \beta g(x))\ dx = \alpha \int_a^b f(x)\ dx + \beta \int_a^b g(x)\ dx.
    \]
  \end{lem}
  
  \begin{prf}
    Given $\varepsilon > 0$, pick $\delta > 0$ such that $|| \Pi || < \delta$ implies 
    \[
      \left | R(f, \Pi) - \int_a^b f(x)\ dx \right | < \varepsilon \land \left | R(g, \Pi) - \int_a^b g(x)\ dx \right | < \varepsilon.
    \]
    
    Since $R(\alpha f + \beta g, \Pi) = \alpha R(f, \Pi) + \beta R(g, \Pi)$, 
    \begin{align*}
      &\left | R(\alpha f + \beta g, \Pi) - \alpha \int_a^b f(x)\ dx - \beta \int_a^b g(x)\ dx \leq \left | \alpha \right | \right | \\ 
      &\quad \leq \left | \alpha \right | \left | R(f, \Pi) - \int_a^b f(x)\ dx \right | + \left | \beta \right | \left | R(g, \Pi) - \int_a^b g(x)\ dx \right | \\ 
      &\quad \leq (\left | \alpha \right | + \left | \beta \right |) \varepsilon.
    \end{align*}
  \end{prf}
  
  \begin{cor}
    Let $f, g: [0, \infty) \to \mathbb R$ be continuous on $[0, \infty)$ and differentiable on $(0, \infty)$.
    Then 
    \[
      f(0) \leq g(0) \land \forall \, x \in (0, \infty): f'(x) \leq g'(x) \Rightarrow \forall \, x \in [0, \infty]: f(x) \leq g(x).
    \]
  \end{cor}
  
  \begin{eg}
    $\forall \, x \geq 0: e^x \geq 1 + x$.
  \end{eg}
  
  \begin{lem}
    Let $f \colon [a, b] \to \mathbb R$ be continuous on $[a, b]$ and differentiable on $(a, b)$.
    Then $f'$ has the intermediate value property.
  \end{lem}
  
  \newpage
  
  \begin{prf}
    Without loss of generality assume $f'$ exists on $[\tilde a, \tilde b]$ such that $\tilde a < a < b < \tilde b$.
    Without loss of generality assume $f'(a) < f'(b)$.
    Let $t \in (f'(a), f'(b))$.
    Let $h(x) := f(x) - t x$.
    Then 
    \[
      h'(a) < 0 \Rightarrow \exists \, x \in (a, b): h(x) < h(a).
    \]
    With the same reasoning, we have 
    \[
      h'(b) > 0 \Rightarrow \exists \, y \in (a, b): h(y) < h(b).
    \]
    Then 
    \[
      \exists \, z \in (a, b) \text{ local minimum} \Rightarrow h'(z) = f(z) - t = 0.
    \]
  \end{prf}
  
  \begin{cor}
    The derivative of a differentiable function does not have discontinuities of first kind.
  \end{cor}
  
  \begin{eg}
    Let 
    \[
      f(x) = \begin{cases}
        x^2 \sin(1 / x) & x \neq 0 \\ 
        0 & x = 0.
      \end{cases}
    \]
    Then $\forall \, x \neq 0: f'(x) = x \sin(1 / x) - \cos(1 / x)$.
    $\lim_{x \to 0^\pm} f'(x)$ does not exist.

    Also note that 
    \[
      \frac{f(x) - f(0)}{x - 0} = x \sin(1 / x) \underset{x \to 0}{\longrightarrow} 0
    \]
    so $f'(0) = 0$.
  \end{eg}
  
  \begin{thm}[L'Hopital's rule, proved by Bernoulli 1694]
    Let $f, g \colon \mathbb R \to \mathbb R$ be continuous and differentiable on $(a - \delta, a + \delta)$ where $a \in \mathbb R$ and $\delta > 0$.
    Assume 
    \[
      f(a) = 0 = g(a) \land \forall \, x \in (a - \delta, a + \delta) \setminus \left \{ a \right \}: g(x) \neq 0 \land g'(x) \neq 0.
    \]
    Theni
    \[
      \lim_{x \to a} \frac{f'(x)}{g'(x)} \text{ exists} \Rightarrow \lim_{x \to a} \frac{f(x)}{g(x)} \text{ exists} \land \lim_{x \to a} \frac{f(x)}{g(x)} = \lim_{x \to a} \frac{f'(x)}{g'(x)}.
    \]
  \end{thm}
  
  \begin{prf}
    Let $x \in (a - \delta, a + \delta) \setminus \left \{ a \right \}$.
    Then for $x > a$ we have 
    \[
      \frac{f(x)}{g(x)} \underset{f(a) = 0, g(a) = 0}{=\joinrel=\joinrel=\joinrel=\joinrel=\joinrel=\joinrel=\joinrel=\joinrel=\joinrel=} \frac{f(x) - f(a)}{g(x) - g(a)} \overset{\exists \, z_x \in (a, x)}{\underset{\text{Cauchy MVT}}{=\joinrel=\joinrel=\joinrel=\joinrel=\joinrel=\joinrel=\joinrel=\joinrel=\joinrel=}} \frac{f'(z_x)}{g'(z_x)}.
    \]
    
    Since $x \to a$ implies $z_x \to a$, existence of $\lim_{z \to a} \frac{f'(z)}{g'(z)}$ gives 
    \[
      \lim_{x \to a} \frac{f(x)}{g(x)} = \lim_{z \to a} \frac{f'(z)}{g'(z)}.
    \]
  \end{prf}
  
  \begin{eg}
    $\lim_{x \to 0} \frac{\sin x}{x} = \lim_{x \to 0} \frac{cos x}{1} = 1$.
  \end{eg}
  
  \section{4.11 Monday Week 3}

  {\boldmath \bfseries Last time:} $f \colon [a, b] \to \mathbb R$ is Riemann integrable (RI) if 
  \[
    \exists \, L \in \mathbb R \, \forall \, \varepsilon > 0 \, \exists \, \delta > 0 \, \forall \, \Pi = \text{marked partition of $[a, b]$}: || \Pi || < \delta \Rightarrow \left | R(f, \Pi) - L \right | < \varepsilon.
  \]
  
  Notation: $L = \int_a^b f(x)\ dx$.
  
  We proved {\boldmath \bfseries linearity}: 
  \[
    \int_a^b (\alpha f(x) + \beta f(x))\ dx = \alpha \int_a^b f(x)\ dx + \beta \int_a^b g(x)\ dx.
  \]
  
  \begin{lem}
    If $f$ is RI on $[a, b]$ then $f$ is bounded on $[a, b]$.
  \end{lem}
  
  \begin{prf}
    RI $\Rightarrow \exists \, \delta > 0 \, \forall \, \Pi = \text{marked partition}: R(f, \Pi) \leq L + 1$.
    Then $\forall \, i = 1, \dots, n \, \forall \, \tilde t_i: f(\tilde t_i) (t_i - t_{i - 1}) + \sum_{j = 1, \dots, n, j \neq i} f(t_j^*) (t_j - t_{j - 1}) \leq L + 1$, which means $\sup_{\tilde t_i \in [t_{i - 1}, t_i]} f(\tilde t_i) < \infty$.
    Then $\sup_{x \in [a, b]} f(x) < \infty$.
  \end{prf}
  
  \begin{lem}[Additivity]
    Let $a < c < b$ be reals.
    If $f$ is RI on $[a, c]$ and on $[c, b]$, then it is RI on $[a, b]$ and 
    \[
      \int_a^b f(x)\ dx = \int_a^c f(x)\ dx + \int_c^b f(x)\ dx.
    \]
  \end{lem}
  
  \begin{prf}
    Let $\varepsilon > 0$ and let $\delta > 0$ be such that $\forall \, \Pi = \text{marked partition of $[a, c]$}$ and \\ $\forall \, \Pi' = \text{marked partition of $[c, b]$}$ such that $||\Pi|| < \delta \land ||\Pi'|| < \delta$ we have 
    \[
      \left | R(f, \Pi) - \int_a^c f(x)\ dx \right | < \varepsilon \quad \land \quad \left | R(f, \Pi') - \int_c^b f(x)\ dx \right | < \varepsilon.
    \]
    
    If $\tilde \Pi$ is a marked partition of $[a, b]$ with $||\tilde \Pi|| < \delta$ containing $c$ then 
    \[
      \left | R(f, \Pi) - \int_a^c f(x)\ dx - \int_c^b f(x)\ dx \right | < 2 \varepsilon.
    \]
    
    Suppose $\tilde \Pi$ does not contain $c$.
    Then adding $c$ to $\tilde \Pi$ changes $R(f, \tilde \Pi)$ by at most $2 \cdot 3 \delta \sup_{x \in [a, b]} \left | f(x) \right |$.
  \end{prf}
  
  \begin{lem}
    If $f$ is RI on $[a, b]$ then 
    \[
      \left | \int_a^b f(x)\ dx \right | \leq (b - a) \underbrace{\sup_{x \in [a, b]} \left | f(x) \right |}_{{} < \infty}.
    \]
  \end{lem}
  
  \begin{prf}
    Note that 
    \[
      \left | R(f, \Pi) \right | = \left | \sum_{i = 1}^n f(t^*_i) (t_i - t_{i - 1}) \right | \leq \sum_{i = 1}^n \left | f(t^*_i) (t_i - t_{i - 1}) \right | = R(\left | f \right |, \Pi) \leq \sup_{x \in [a, b]} \left | f(x) \right | \underbrace{\sum_{i = 1}^n (t_i - t_{i - 1})}_{{} = b - a}
    \]
  \end{prf}
  
  \begin{note}
    If we knew that $\left | f \right |$ is RI, then this gives 
    \[
      \left | \int_a^b f(x)\ dx \right | \leq \int_a^b \left | f(x) \right |\ dx.
    \]
  \end{note}
  
  {\boldmath \bfseries Q:} Sufficient conditions for RI?
  
  {\boldmath \bfseries A:} We will answer this using Darboux's version of Riemann integral.
  
  \begin{defn}
    Let $f \colon [a, b] \to \mathbb R$ be bounded.
    Given an unmarked partition $\Pi = \left \{ t_i \right \}_{i = 1}^n$ of $[a, b]$, set 
    \[
      U(f, \Pi) := \sum_{i = 1}^n \sup \left \{ f(x) : x \in [t_{i - 1}, t_i] \right \} (t_i - t_{i - 1})
    \]
    and 
    \[
      L(f, \Pi) := \sum_{i = 1}^n \inf \left \{ f(x) : x \in [t_{i - 1}, t_i] \right \} (t_i - t_{i - 1})
    \]
    to be the {\boldmath \bfseries upper and lower Darboux sums}.
  \end{defn}
  
  \begin{note}
    $L(f, \Pi) \leq R(f, \Pi) \leq U(f, \Pi)$ for any marked partition $\Pi$.
  \end{note}
  
  \begin{lem}
    For all unmarked partitions $\Pi$ and $\Pi'$ of $[a, b]$ we have 
    \[
      L(f, \Pi) \leq U(f, \Pi').
    \]
  \end{lem}
  
  \begin{prf}
    Assume first $\Pi$ is a subset of $\Pi'$, meaning that all points of $\Pi$ are included in $\Pi'$.
    We claim that $U(f, \Pi') \leq U(f, \Pi)$ and $L(f, \Pi') \geq L(f, \Pi)$.
    
    Note that if $\Pi' = \Pi \cup \left \{ t \right \}$, let $[t_{i - 1}, t_i]$ be the interval containing $t$.
    Then 
    \[
      \max \left \{ \sup_{x \in [t_{i - 1}, t]} f(x), \sup_{x \in [t, t_i]} f(x) \right \} \sup_{x \in [t_{i - 1}, t_i]} f(x), 
    \]
    resulting in $U(f, \Pi') \leq U(f, \Pi)$.

    Now let $\Pi$ and $\Pi'$ be arbitrary and $\Pi \cup \Pi'$ be their common refinement.
    Then 
    \[
      L(f, \Pi) \leq L(f, \Pi \cup \Pi') \leq U(f, \Pi \cup \Pi') \leq U(f, \Pi').
    \]
  \end{prf}
  
  \begin{defn}
    Set 
    \[
      \underline{\int_a^b} f(x)\ dx := \sup \left \{ L(f, \Pi) : \Pi = \text{partition of $[a, b]$} \right \}
    \]
    and 
    \[
      \overline{\int_a^b} f(x)\ dx := \inf \left \{ U(f, \Pi) : \Pi = \text{partition of $[a, b]$} \right \}
    \]
    to be the {\boldmath \bfseries lower and upper Darboux integrals}.
  \end{defn}
  
  \begin{note}
    \[
      \underline{\int_a^b} f(x)\ dx \leq \overline{\int_a^b} f(x)\ dx.
    \]
  \end{note}
  
  \begin{defn}
    We say that a bounded $f$ is {\boldmath \bfseries Darboux integrable on $[a, b]$} if 
    \[
      \underline{\int_a^b} f(x)\ dx = \overline{\int_a^b} f(x)\ dx.
    \]
  \end{defn}
  
  \section{4.13 Wednesday Week 3}
  
  {\boldmath \bfseries \underline{Riemann integral continued}}

  {\boldmath \bfseries Last time:} $U(f, \Pi)$ and $L(f, \Pi)$ are the upper and lower Darboux sums.
  Note that 
  \[
    \forall \, \Pi, \Pi': L(f, \Pi) \leq U(f, \Pi').
  \]
  
  Then 
  \[
    \overline{\int_a^b} f(x)\ dx = \inf \left \{ U(f, \Pi) : \text{$\Pi$ partition} \right \}
  \]
  and 
  \[
    \underline{\int_a^b} f(x)\ dx = \sup \left \{ L(f, \Pi) : \text{$\Pi$ partition} \right \}
  \]
  obey
  \[
    \underline{\int_a^b} f(x)\ dx \leq \overline{\int_a^b} f(x)\ dx.
  \]
  
  \newpage
  
  \begin{defn}
    $f \colon [a, b] \to \mathbb R$ bounded is {\boldmath \bfseries Darboux integrable} if 
    \[
      \underline{\int_a^b} f(x)\ dx = \overline{\int_a^b} f(x)\ dx.
    \]
  \end{defn}
  
  \begin{lem}
    For every $f \colon [a, b] \to \mathbb R$: 
    \[
      \text{$f$ Darboux integrable} \Leftrightarrow \forall \, \varepsilon > 0 \, \exists \, \Pi \text{ partition}: U(f, \Pi) - L(f, \Pi) < \varepsilon.
    \]
  \end{lem}
  
  \begin{prf}
    By definition, 
    \[
      \forall \, \varepsilon > 0 \, \exists \, \Pi, \tilde \Pi: U(f, \Pi) < \overline{\int_a^b} f(x)\ dx + \varepsilon \quad \land \quad L(f, \tilde \Pi) > \underline{\int_a^b} f(x)\ dx - \varepsilon.
    \]
    
    Then 
    \[
      U(f, \Pi \cup \tilde \Pi) - L(f, \Pi \cup \tilde \Pi) \leq U(f, \Pi) - L(f, \tilde \Pi) \leq \overline{\int_a^b} f(x)\ dx - \underline{\int_a^b} f(x)\ dx + 2 \varepsilon.
    \]
    
    Then the equality of the Darboux integrals implies the left to right direction of the lemma.
    
    For the converse, 
    \[
      0 \leq \overline{\int_a^b} f(x)\ dx - \underline{\int_a^b} f(x)\ dx \leq U(f, \Pi) - L(f, \Pi) < \varepsilon.
    \]
  \end{prf}
  
  \begin{lem}
    Let $\Pi$ and $\Pi'$ be unmarked partitions.
    Then 
    \[
      U(f, \Pi') \geq U(f, \Pi) - 2 \left | \Pi' \right | ||\Pi|| ||f||
    \]
    and 
    \[
      L(f, \Pi') \leq L(f, \Pi) + 2 \left | \Pi' \right | ||\Pi|| ||f||.
    \]
    where $||f|| := \sup_{x \in [a, b]} \left | f(x) \right |$.
  \end{lem}
  
  \begin{prf}
    Note that 
    \[
      U(f, \Pi') \geq U(f, \Pi \cup \Pi')
    \]
    and for $f \geq 0$, dropping intervals of $\Pi$ that receive points in $\Pi'$ from $U(f, \Pi)$ changes the result by at most $2 \left | \Pi' \right | ||\Pi|| ||f||$.
  \end{prf}
  
  \newpage
  
  \begin{thm}
    For every $f \colon [a, b] \to \mathbb R$ bounded: 
    \[
      \text{$f$ Riemann integrable} \Leftrightarrow \text{$f$ Darboux integrable}.
    \]
    
    If both are true then 
    \[
      \int_a^b f(x)\ dx = \underline{\int_a^b} f(x)\ dx = \overline{\int_a^b} f(x)\ dx.
    \]
  \end{thm}
  
  \begin{prf}
    $\Rightarrow$: RI means that 
    \[
      \exists \, L \in \mathbb R \, \exists \, \delta > 0 \, \forall \, \Pi \text{ partition with $||\Pi|| < \delta$}: \left | R(f, \Pi) - L \right | < \varepsilon.
    \]
    
    Pick $N \in \mathbb N$ such that $N > (b - a) / \delta$, define $\Pi = \left \{ t_i \right \}_{i = 1}^n$ such that $t_i - t_{i - 1} = \frac{b - a}{N} < \delta$. 
    Now pick $t^*_i \in [t_{i - 1}, t_i]$ such that 
    \[
      f(t^*_i) \geq \sup \left \{ f(x) : x \in [t_{i - 1}, t_i] \right \} - \frac{\varepsilon}{b - a}
    \]
    and $\tilde{t^*_i} \in [t_{i - 1}, t_i]$ such that 
    \[
      f(\tilde{t^*_i}) \leq \inf \left \{ f(x) : x \in [t_{i - 1}, t_i] \right \} + \frac{\varepsilon}{b - a}.
    \]
    Then let $\Pi$ be the partition with marked points $\left \{ t^*_i \right \}_{i = 1}^N$ and $\tilde \Pi$ be the partition with marked points $\left \{ \tilde{t^*_i} \right \}_{i = 1}^N$.

    Then 
    \[
      U(f, \Pi) \leq \sum_{i = 1}^N \left ( f(t^*_i) + \frac{\varepsilon}{b - a} \right ) (t_i - t_{i - 1}) = R(f, \Pi) + \varepsilon
    \]
    and 
    \[
      L(f, \tilde \Pi) \geq \sum_{i = 1}^n \left ( f(\tilde{t^*_i} - \frac{\varepsilon}{b - a}) \right ) (t_i - t_{i - 1}) = R(f, \tilde \Pi) - \varepsilon.
    \]
    Now 
    \begin{align*}
      U(f, \Pi \cup \tilde \Pi) - L(f, \Pi \cup \tilde \Pi) &\leq U(f, \Pi) - L(f, \tilde \Pi) \\ 
      &\leq R(f, \Pi) - R(f, \tilde \Pi) + 2 \varepsilon \\ 
      &\leq \left | R(f, \Pi) - L \right | + \left | R(f, \tilde \Pi) - L \right | + 2 \varepsilon \\ 
      &\leq 4 \varepsilon.
    \end{align*}
    
    $\Leftarrow$: $\forall \, \varepsilon > 0 \exists \, \Pi' \text{ partition such that } U(f, \Pi') - L(f, \Pi') < \varepsilon$.
    Pick any $\Pi$ and $\tilde \Pi$ marked partitions with $||\tilde \Pi||, ||\Pi|| < \delta := \varepsilon / (\left | \Pi' \right | ||f||) \quad (f \neq 0)$.
    
    Then 
    \[
      R(f, \Pi) \leq U(f, \Pi) \overset{\text{by Lemma~\ref{lem:10.3}}}{\leq} U(f, \Pi') + 2 \underbrace{\left | \Pi' \right | ||\Pi|| ||f||}_{\leq \varepsilon}
    \]
    and 
    \[
      R(f, \tilde \Pi) \geq L(f, \tilde \Pi) \overset{\text{by Lemma~\ref{lem:10.3}}}{\geq} L(f, \Pi') - 2 \underbrace{\left | \Pi' \right | ||\tilde \Pi|| ||f||}_{\leq \varepsilon}.
    \]
    
    Then 
    \[
      \left | R(f, \tilde \Pi) - R(f, \tilde \Pi) \right | \leq U(f, \Pi') - L(f, \Pi') + 4 \varepsilon \leq 5 \varepsilon.
    \]
    
    Let $\left \{ \Pi_n \right \}$ be an arbitrary sequence of marked partitions such that 
    \[
      ||\Pi_n|| \to 0 \quad \land \quad L := \lim_{n \to \infty} R(f, \Pi_n) \text{ exists}.
    \]
    This exists by Bolzano-Weierstrass theorem.
    
    Then 
    \[
      \left | R(f, \Pi) - L \right | \leq \left | R(f, \Pi_n) - L \right | + \left | R(f, \Pi) - R(f, \Pi_n) \right | \underset{\text{once } ||\Pi_n|| < \delta}{\leq} \left | R(f, \Pi_n) - L \right | + 5 \varepsilon \underset{n \to \infty}{\longrightarrow} 5 \varepsilon.
    \]
    
    Then we showed that 
    \[
      \exists \, L \in \mathbb R \, \forall \, \varepsilon > 0 \, \exists \, \delta > 0 \, \forall \, \Pi \text{ marked partition}: ||\Pi|| < \delta \Rightarrow \left | R(f, \Pi) - L \right | \leq 5 \varepsilon.
    \]
  \end{prf}
  
  \begin{cor}
    Let $f \colon [a, b] \to \mathbb R$ be bounded.
    Then 
    \[
      \text{$f$ is RI} \Leftrightarrow \forall \, \varepsilon > 0 \, \exists \, \Pi = \left \{ t_i \right \}_{i = 1}^n \text{ unmarked partition}: \sum_{i = 1}^N \operatorname{osc}(f, [t_{i - 1}, t_i])(t_i - t_{i - 1}) < \varepsilon
    \]
    where $\operatorname{osc}(f, A) = \sup_{x \in A} f(x) - \inf_{x \in A} f(x)$
  \end{cor}
  
  \begin{eg}
    The dirichlet function 
    \[
      f(x) = \begin{cases}
        1 & x \in \mathbb Q \\ 
        0 & x \not \in \mathbb Q
      \end{cases}
    \]
    is not RI.
  \end{eg}
  
  \begin{eg}
    The function 
    \[
      f(x) = \begin{cases}
        \frac{1}{n + 1} & x = q_n \\ 
        0 & x \not \in \mathbb Q
      \end{cases}
    \]
    is RI.
  \end{eg}
  
  \section{4.15 Friday Week 3}
  
  {\boldmath \bfseries \underline{Riemann Integrability - criteria and characterization}}
  
  {\boldmath \bfseries Last time:} $\forall \, f \colon [a, b] \to \mathbb R$ bounded, 
  \[
    \text{$f$ RI} \Leftrightarrow \forall \, \varepsilon > 0 \, \exists \, \Pi = \left \{ t_i \right \}_{i = 1}^n \text{ partition of $[a, b]$}: \sum_{i = 1}^n \operatorname{osc}(f, [t_{i - 1}, t_i]) \left | t_i - t_{i - 1} \right | < \varepsilon
  \]
  where 
  \begin{align*}
    \operatorname{osc}(f, A) &:= \sup \left \{ \left | f(y) - f(x) \right | : x, y \in A \right \} \\ 
    &= \sup_{x \in A} f(x) - \inf_{x \in A} f(x) (A \neq \varnothing).
  \end{align*}
  
  \begin{lem}
    Let $f, g \colon [a, b] \to \mathbb R$.
    Then 
    \begin{enumerate}
      \item $\text{$f$ RI} \Rightarrow \text{$\left | f \right |$ RI}$ and 

      \item $\text{$f$, $g$ RI} \Rightarrow \text{$f \cdot g$ RI}$.
    \end{enumerate}
  \end{lem}
  
  \begin{prf}
    Note that 
    \[
      \left | \left | f \right |(x) - \left | f \right |(y) \right | = \left | \left | f(x) \right | - \left | f(y) \right | \right | \leq \left | f(x) - f(y) \right |.
    \]
    Then 
    \[
      \operatorname{osc}(f, A) \leq \operatorname{osc}(\left | f \right |, A).
    \]
    Then 
    \[
      \text{$f$ RI} \Rightarrow \text{$\left | f \right |$ RI}.
    \]
    Note that a counterexample for the converse is Dirichlet's function.
  \end{prf}
  
  \begin{thm}
    For all $f \colon [a, b] \to \mathbb R$ we have 
    \[
      \text{$f$ continuous} \Rightarrow \text{$f$ RI}.
    \]
  \end{thm}
  
  \begin{prf}
    Note that $[a, b]$ compact and $f$ continuous implies that $f$ is uniformly continuous.
    Then for all $\varepsilon > 0$ there exists $\delta > 0$ such that for all $s, t \in [a, b]$ we have 
    \[
      0 < \left | s - t \right | < \delta \Rightarrow \operatorname{osc}(f, [s, t]) < \frac{\varepsilon}{b - a}.
    \]
    Then for all 
    \[
      \forall \, \Pi: ||\Pi|| < \delta \Rightarrow \sum_{i = 1}^n \operatorname{osc}(f, [t_{i - 1}, t_i]) (t_i - t_{i - 1}) \leq \sum_{i = 1}^n \frac{\varepsilon}{b - a} (t_i - t_{i - 1}) \leq \varepsilon.
    \]
  \end{prf}
  
  \begin{lem}
    Let $f \colon [a, b] \to \mathbb R$ be bounded and such that $f$ has only finitely many discontinuities.
    Then $f$ is RI.
  \end{lem}
  
  \begin{prf}
    Let $x_1, \dots, x_m$ enumerate discontinuity points of $f$.
    Pick $\varepsilon > 0$.
    Suppose without loss of generality $||f|| \neq 0$.
    Let $\delta < \frac{\varepsilon}{m ||f||}$.
    Then 
    \[
      \operatorname{osc}(f, [x_i - \delta, x_i + \delta] \cap [a, b]) \leq 2 ||f||.
    \]
    Next, note that $[a, b] \setminus \bigcup_{i = 1}^m (x_i - \delta, x_i + \delta)$ is closed and thus compact.
    Then $f$ is uniformly continuous on this set.
    Then there exists $\delta' > 0$ such that for all $[s, t] \subseteq $ this set we have 
    \[
      0 < \left | s - t \right | \leq \delta' \Rightarrow \operatorname{osc}(f, [s, t]) \leq \frac{\varepsilon}{b - a}.
    \]
    Now partition $[a, b] \setminus \bigcup_{i = 1}^n (x_i - \delta, x_i + \delta)$ into intervals of length $\leq \delta'$.
    Combine them with intervals $[x_i - \delta, x_i + \delta]$.
    Now take $\Pi = $ set of endpoints of these intervals.
    Then 
    \[
      \sum_{i = 1}^n \operatorname{osc}(f, [t_{i - 1}, t_i]) (t_i - t_{i - 1}) \leq m \cdot 2 ||f|| \cdot 2 \delta + \frac{\varepsilon}{b - a} (b - a) \leq 5 \varepsilon.
    \]
  \end{prf}
  
  \begin{lem}
    Let $f \colon [a, b] \to \mathbb R$ be bounded.
    \[
      \text{$f$ has no discontinuities of second kind} \Rightarrow \text{$f$ RI}.
    \]
  \end{lem}
  
  \begin{prf}
    Key idea: 
    \[
      \forall \, \eta > 0: \left \{ x \in (a, b) : \operatorname{diam}\{ \lim_{z \to x^+} f(z), \lim_{z \to x^-} f(z), f(x) \} > \eta \right \} \text{ is finite}.
    \]
  \end{prf}
  
  \begin{eg}
    \[
      f(x) = \begin{cases}
        \frac{1}{n + 1} & x = q_n \\ 
        0 & x \not \in \mathbb Q.
      \end{cases}
    \]
  \end{eg}
  
  {\boldmath \bfseries It gets worse:} Let 
  \[ 
    C := \left \{ \sum_{i \in \mathbb N} \frac{2 \sigma_i}{3^{i + 1}} : \left \{ \sigma_i \right \}_{i \in \mathbb N} \in \left \{ 0, 1 \right \}^\mathbb N \right \}
  \]
  be Cantor's ternary set.

  Then $C = \bigcap_{n \in \mathbb N} C_n$ where 
  \[
    C_n = \left \{ \sum_{i = 1}^n \frac{\sigma_i}{3^{i + 1}} + [0, 3^{-n - 1}] : \sigma_1, \dots, \sigma_n \left \{ 0, 1 \right \} \right \}.
  \]
  
  \begin{lem}
    The function 
    \[
      1_C(x) = \begin{cases}
        1 & x \in C \\ 
        0 & x \not \in C
      \end{cases}
    \]
    is RI.
  \end{lem}
  
  \begin{prf}
    Let $I_1, \dots, I_{2^n}$ be intervals constituting $C_n$.
    Define 
    \[
      J_k = \left \{ x \in [0, 1] : \operatorname{dist}(x, I_k) < \frac{1}{3^{n + 1}} \right \}.
    \]
    Then 
    \[
      \operatorname{length}(J_k) = \operatorname{length}(I_k) + 2 \cdot \frac{1}{3^{n + 1}} \leq \frac{1}{3^n}.
    \] 
    Take $\Pi$ to be the endpoints of $\left \{ J_k \right \}_{k = 1}^{2^n}$.
    Then 
    \[
      \sum_{i = 1}^m \operatorname{osc}(f, [t_{i - 1}, t_i]) \left | t_i - t_{i - 1} \right | \leq \sum_{k = 1}^{2^n} \operatorname{length}(J_k) \leq 2^n \cdot \frac{1}{3^n} \underset{n \to \infty}{\longrightarrow} 0.
    \]
  \end{prf}
  
  \section{4.18 Monday Week 4}
  
  {\boldmath \bfseries \underline{Characterizing Riemann integrability}}
  
  Sufficient conditions for RI: continuity, finite number of discontinuities, no discontinuities of second kind.

  Necessary condition for RI: boundedness.
  
  \begin{defn}
    A set $A \subseteq \mathbb R$ is of {\boldmath \bfseries zero length} if 
    \[
      \forall \, \varepsilon > 0 \, \exists \, \left \{ (a_i, b_i) \right \}_{i \in \mathbb N} \text{ intervals}: A \leq \bigcup_{i \in \mathbb N} (a_i, b_i) \quad \land \quad \sum_{i \in \mathbb N} (b_i - a_i) < \varepsilon.
    \]
  \end{defn}
  
  \begin{lem}
    In the definition of zero length, closed intervals can be used.
  \end{lem}
  
  \begin{prf}
    If $A \subseteq \bigcup_{i \in \mathbb N} [a_i, b_i]$, let $\tilde{a_i} = a_i - \varepsilon / 2^i$ and $\tilde{b_i} = b_i + \varepsilon / 2^i$.
    Then 
    \[
      A \subseteq \bigcup_{i \in \mathbb N} (\tilde{a_i}, \tilde{b_i})
    \]
    and 
    \[
      \sum_{i \in \mathbb N} (\tilde{b_i} - \tilde{a_i}) = \sum_{i \in \mathbb N} (b_i - a_i) + \sum_{i \in \mathbb N} 2 \cdot \frac{\varepsilon}{2^i} = \sum_{i \in \mathbb N} (b_i - a_i) + 4 \varepsilon.
    \]
  \end{prf}
  
  \begin{lem}
    Let $f \colon \mathbb R \to \mathbb R$ be bounded.
    Set 
    \[ 
      M_f(x) = \inf_{\delta > 0} \sup_{z : \left | z - x \right | < \delta} f(z)
    \]
    and 
    \[
      m_f(x) = \sup_{\delta > 0} \inf_{z : \left | z - x \right | < \delta} f(z).
    \]
    Then 
    \begin{enumerate}
      \item $\forall \, x \in \mathbb R: \text{$f$ continuous at $x$} \Leftrightarrow M_f(x) = m_f(x)$, 
        
      \item $\forall \, x \in \mathbb R \, \forall \, \delta: \max \left \{ \operatorname{osc}(f, [x - \delta, x]), \operatorname{osc}(f, [x - \delta, x]) \right \} \geq M_f(x) - m_f(x)$, and 
        
      \item $\forall \, x \in \mathbb R: \lim_{\delta \to 0} \operatorname{osc}(f, [x - \delta, x + \delta]) = M_f(x) - m_f(x)$.
    \end{enumerate}
  \end{lem}
  
  \begin{thm}[Lebesgue's characterization of Riemann integrability]
    Let $f \colon [a, b] \to \mathbb R$ be bounded.
    Then 
    \[
      f \text{ RI} \Leftrightarrow \left \{ x \in [a, b] : \text{$f$ discontinuous at $x$} \right \} \text{ is zero length}.
    \]
  \end{thm}
  
  \begin{prf}
    $\Rightarrow$: Let $f \colon [a, b] \to \mathbb R$ be bounded and RI.
    
    Pick $\varepsilon > 0$.
    Then RI implies 
    \[
      \forall \, n \in \mathbb N \, \exists \, \Pi = \left \{ t_i^n \right \}_{i = 1}^{m(n)} \text{ partition of $[a, b]$}: \sum_{i = 1}^{m(n)} \operatorname{osc}(f, [t_{i - 1}^n, t_i^n]) (t_i^n - t_{i - 1}^n) < \varepsilon 4^{-n}.
    \]
    Set $I_n := \left \{ i = 1, \dots, m(n) : \operatorname{osc}(f, [t_{i - 1}^n, t_i^n]) > 2^{-n} \right \}$.
    Then 
    \[
      \sum_{i \in I_n} (t_i^n - t_{i - 1}^n) \overset{\text{Markov's inequality}}{\leq} \sum_{i \in I_n} \frac{\operatorname{osc}(f, [t_{i - 1}^n, t_i^n])}{2^{-n}} (t_i^n - t_{i - 1}^n) \leq 2^n \sum_{i = 1}^{m(n)} \operatorname{osc}(f, [t_{i - 1}^n, t_i^n]) (t_i^n - t_{i - 1}^n) \leq 2^n \cdot 4^{-n} = \varepsilon 2^{-n}.
    \]
    Now 
    \[
      \left \{ x \in [a, b] : M_f(x) \neq m_f(x) \right \} \subseteq \bigcup_{n \geq 1} \bigcup_{i \in I_n} [t_{i - 1}^n, t_i^n].
    \]
    Then  
    \[
      \sum_{n \geq 1} \sum_{i \in I_n} (t_i^n - t_{i - 1}^n) \leq \sum_{n \geq 1} \varepsilon 2^{-n} = \varepsilon.
    \]
    Then $f$ RI $\Rightarrow$ $\left \{ x \in [a, b] : M_f(x) \neq m_f(x) \right \}$ is zero length.
    
    $\Leftarrow$: Let $\varepsilon > 0$ and let $\left \{ J_i \right \}_{i \in \mathbb N}$ be open intervals such that 
    \[
      \left \{ x \in [a, b] : M_f(x) \neq m_f(x) \right \} \subseteq \bigcup_{i \in \mathbb N} J_i \quad \land \quad \sum_{ i \in \mathbb N} \operatorname{length}(J) < \frac{\varepsilon}{2 \varepsilon ||f||} (f \neq 0).
    \]
    Since $M_f(x) = m_f(x) \Rightarrow x$ is continuous: 
    \[
      \forall \, x \in [a, b]: M_f(x) = m_f(x) \Rightarrow \exists \, \delta_x > 0: \operatorname{osc}(f, (x - \delta x, x + \delta x)) < \frac{\varepsilon}{b - a}.
    \]
    Then intervals $\left \{ J_i \right \}_{i \in \mathbb N} \cup \left \{ (x - \delta, x + \delta) : M_f(x) = m_f(x) \right \}$ cover $[a, b]$.
    Then by Heine-Borel theorem, 
    \[
      \exists \, m, n \in \mathbb N \, \exists \, x_0, \dots, x_m \in \left \{ x \in [a, b] : M_f(x) = m_f(x)) \right \}: [a, b] \subseteq \bigcup_{i = 0} J_i \cup \bigcup_{j = 0}^m (x_j - \delta_{x_j}, x_j + \delta_{x_j}).
    \]
    Let $\Pi = \left \{ t_i \right \}_{i = 1}^N$ be a partition containing of all endpoints of the intervals $(x_j - \delta_{x_j}, x_j + \delta_{x_j})$.
    Let $k = \left \{ i = 1, \dots, N : [t_{i - 1}, t_i] \subseteq \bigcup_{j = 1}^m (x_j - \delta_{x_j}, x_j + \delta_{x_j}) \right \}$.
    Then 
    \[
      \forall \, i \in K: \operatorname{osc}(f, [t_{i - 1}, t_i]) < \frac{\varepsilon}{b - a}
    \]
    and 
    \[
      \sum_{i \not \in K} \operatorname{osc}(f, [t_{i - 1}, t_i]) \leq 2 ||f|| \cdot \sum_{i \not \in K} (t_i - t_{i - 1}) < 2 ||f|| \sum_{i \in \mathbb N} \operatorname{length}(J_i) < \varepsilon.
    \]
    Then 
    \[
      \sum_{i = 1}^n \operatorname{osc}(f, [t_{i - 1}, t_i]) (t_i - t_{i - 1}) \leq \sum_{i \in K} \operatorname{osc}(f, [t_{i - 1}, t_i]) (t_i - t_{i - 1}) + \sum_{i \not \in K} \operatorname{osc}(f, [t_{i - 1}, t_i]) (t_i - t_{i - 1}) \leq \frac{\varepsilon}{b - a} (b - a) + \varepsilon = 2 \varepsilon. 
    \]
  \end{prf}
  
  \section{4.20 Wednesday Week 4}

  {\boldmath \bfseries \underline{Derivative vs. integral, FTC, \dots}}

  {\boldmath \bfseries Last time:} $\text{$f$ RI} \Leftrightarrow \left \{ x \in [a, b] : \text{$f$ discnotinuous at $x$} \right \} \text{ is of zero length}$.
  
  \begin{cor}
    \[
      \text{f RI} \land \left \{ x \in [a, b] : g(x) \neq f(x) \right \} \text{ is of zero length} \Rightarrow \text{$g$ RI} \land \int_a^b g(x)\ dx = \int_a^b f(x)\ dx.
    \]
  \end{cor}
  
  {\boldmath \bfseries Today:} Newton / Leibniz FTC: 
  \[
    \frac{d}{dx} \int_a^x f(t)\ dt = f(x) \quad \land \quad \int_a^b \frac{d}{dx} f(t)\ dt = f(b) - f(a).
  \]
  
  \begin{note}
    These are not true without conditions.
  \end{note}
  
  \begin{lem}
    Let $a < b$ be reals and $f \colon [a, b] \to \mathbb R$ be an RI function on $[a, b]$.
    Set $F(x) = \int_a^x f(t)\ dt$.
    Then $F$ is Lipschitz continuous.
  \end{lem}
  
  \begin{prf}
    If $a \leq x < y \leq b$ then additivity implies 
    \[
      F(y) - F(x) = \int_0^y f(t)\ dt - \int_0^x f(t)\ dt = \int_x^y f(t)\ dt.
    \]
    
    Note that $\text{$f$ RI} \Rightarrow \text{$f$ bounded}$.
    Then 
    \[
      \left | F(y) - F(x) \right | = \left | \int_x^y f(t)\ dt \right | \leq ||f|| \cdot \left | y - x \right |.
    \]
  \end{prf}
  
  \begin{eg}
    \[
      \left | x \right | = \int_0^x t (1_{[0, \infty)} - 1_{(-\infty, 0)})\ dt.
    \]
  \end{eg}
  
  {\boldmath \bfseries Q:} Is every Lipschitz function a Riemann integral?
  
  \begin{lem}
    Let $f$ be RI on $[a, b]$.
    Set $F(x) = \int_a^b f(t)\ dt$.
    Then 
    \[
      \forall \, x \in (a, b): \text{$f$ continuous at $x$} \Rightarrow \text{$F'(x)$ exists} \land F'(x) = f(x).
    \]
  \end{lem}
  
  \begin{prf}
    Let $y \in (x, b)$.
    Then 
    \[
      F(y) - F(x) - f(x) (y - x) = \int_x^y (f(t) - f(x))\ dt.
    \]
    Then 
    \[
      \left | \frac{F(y) - F(x)}{y - x} - f(x) \right | \leq \sup_{t \in [x, y]} \left | f(y) - f(x) \right | \underset{y \to x^+}{\longrightarrow} 0 \text{ by right continuity of $f$}.
    \]
    
    Then $F'^+(x) = f(x)$.
    Similarly, $F'^-(x) = f(x)$.
  \end{prf}
  
  \begin{eg}
    \[
      f(x) = 1_{1 / (n + 1), n \in \mathbb N}.
    \]
    Note that $F(x) = 0$ for all $x \in \mathbb R$.
  \end{eg}
  
  \begin{cor}[Fundamental theorem of calculus I]
    Let $f \colon [a, b] \to \mathbb R$ be continuous.
    Then 
    \[
      \forall \, x \in (a, b): \frac{d}{dx} \int_a^x f(t)\ dt = f(x).
    \]
  \end{cor}
  
  \begin{note}
    \begin{itemize}
      \item The integral is an antiderivative / primitive function.
      Notation $\int f(t)\ dt$; 
  
      \item $\frac{d}{dt} \int_x^b f(t)\ dt = - f(x)$; 
      
      \item $\frac{d}{dx} \int_{g(x)}^{h(x)} = f(h(x)) h'(x) - f(g(x)) g'(x)$.
    \end{itemize}
  \end{note}
  
  \begin{thm}
    Let $f \colon [a, b]\to \mathbb R$ be continuous on $[a, b]$ and differentiable on $(a, b)$.
    (Choose $f'(a)$, $'(b)$) arbitrarily.
    Then 
    \[
      \text{ $f$ RI} \Rightarrow \int_a^b f'(t)\ dt = f(b) - f(a).
    \]
  \end{thm}
  
  \begin{prf}
    Let $\varepsilon > 0$.
    Then $f'$ RI implies 
    \[
      \exists \, \delta > 0 \, \forall \, \Pi: ||\Pi|| < \varepsilon \Rightarrow \left | R(f', \Pi) - \int_a^b f'(t)\ dt \right | < \varepsilon.
    \]
    Pick $n \in \mathbb N$ such that $n \delta > (b - a)$.
    Set $t_i := a + \frac{i}{n} (b - a)$ where $i = 0, \dots, n$.

    Then, by the mean value theorem, for all $i = 1, \dots, n$ we have 
    \[
      \exists \, t_i^* \in (t_{i - 1}, t_i): f(t_i) - f(t_{i - 1}) = f'(t_i^*) (t_i - t_{i - 1}).
    \]
    
    Let $\Pi := (\left \{ t_i \right \}, \left \{ t_i^* \right \})$.
    Then 
    \[
      f(b) - f(a) = \sum_{i = 1}^n (f(t_i) - f(t_{i - 1})) = \sum_{i = 1^n} f'(t_i^*) (t_i - t_{i - 1}) = R(f', \Pi).
    \]
    Then 
    \[
      \left | f(b) - f(a) - \int_a^b f'(t)\ dt \right | < \varepsilon.
    \]
  \end{prf}
  
  {\boldmath \bfseries \underline{Volterra's example}}

  \[
    \exists \, F \colon [0, 1] \to \mathbb R \text{ continuous}: \text{$F'(x)$ exists for all $x \in [0, 1]$} \land \text{$F'$ bounded} \land \text{$F'$ is not RI}.
  \]
  
  This is a major deficiency in Riemann's theory that led Lebesgue to the formulation of the Lebesgue integral.
  
  \begin{cor}[Integration by parts]
    Let $f, g \colon [a, b] \to \mathbb R$ be continuous on $[a, b]$ and differentiable on $(a, b)$.
    Then 
    \[
      \text{$f' g$ RI} \land \text{$f g'$ RI} \Rightarrow \int_0^b f'(x) g(x)\ dx = f(b) g(b) - f(a) g(a) - \int_a^b f(x) g'(x)\ dx.
    \]
  \end{cor}
  
  \begin{prf}
    Note that 
    \[
      \text{$f' g$ RI} \land \text{$g' f$ RI} \Rightarrow \text{$(f g)'$ RI}.
    \]
    Then 
    \[
      f(b) g(b) - f(a) g(a) = \int_a^b (f g)'(x)\ dx = \int_a^b f'(x) g(x)\ dx + \int_a^b f(x) g'(x)\ dx.
    \]
  \end{prf}
  
  \begin{cor}
    Let $f \colon [c, d] \to \mathbb R$ and $\varphi \colon [a, b] \to \mathbb R$ be functions.
    Assume 
    \begin{enumerate}
      \item $\varphi$ is continuous on $[a, b]$ and differentiable on $(a, b)$, 

      \item $f$ is continuous on $[c, d]$, and

      \item $(f \circ \varphi) \varphi'$ is RI on $[a, b]$.
    \end{enumerate}
    Then 
    \[
      \int_{\varphi(a)}^{\varphi(b)} f(x)\ dx = \int_a^b (f \circ \varphi)(t) \varphi'(t)\ dt.
    \]
  \end{cor}
  
  \begin{prf}
    Note that 
    \[
      F(x) = \int_a^x f(t)\ dt \overset{\text{FTC I}}{=\joinrel\Longrightarrow} F'(x) = f(x).
    \]
    
    Then 
    \[
      \int_{\varphi(a)}^{\varphi(b)} f(t)\ dt \overset{\text{FTC II}}{=\joinrel=\joinrel=\joinrel=} F(\varphi(b)) - F(\varphi(a)) \overset{\text{FTC II}}{=\joinrel=\joinrel=\joinrel=} \int_a^b \frac{d}{dx} (F \circ \varphi)(x)\ dx = \int_a^b (f \circ \varphi)(x) \varphi'(x)\ dx.
    \]
  \end{prf}
  
  \newpage
  
  \section{4.25 Monday Week 5}
  
  {\boldmath \bfseries \underline{Taylor's theorem}}

  {\boldmath \bfseries Last time:} FTC I: 
  \[
    \text{$f$ continuous} \Rightarrow F(x) = \int_a^x f(t)\ dt \text{ differentiable} \land F'(x) = f(x).
  \]
  
  FTC II: 
  \[
    \text{$F$ continuous on $[a, b]$} \land \text{$F'$ exists on $(a, b)$} \land \text{$F'$ RI} \Rightarrow F(b) - F(a) = \int_a^b  F'(x)\ dx.
  \]
  
  Cantor's function (``Devil's staircase''): 
  \[
    x \in \sum_{i = 0}^n \frac{2 \sigma_i}{3^{i n}} + [0, 3^{-n - 1}] \mapsto F(x) = \sum_{i = b}^n \frac{\sigma_i}{2^{i + 1}}.
  \]
  
  This is simply not an integral of a derivative (not Lipschitz but Holder continuous with a coefficient less than 1).
  $F'$ exists at every point excluding the Cantor set, which is 0.
  
  Consequences of the FTC: 
  \begin{itemize}
    \item Substitution rule

    \item Integration by parts
  \end{itemize}
  
  \begin{thm}[Taylor's theorem with remainder]
    Let $f \colon (a, b) \to \mathbb R$ be $(n + 1)$-times differentiable with $f^{(n + 1)}$ Riemann integrable.
    Then 
    \[
      \forall \, x, x_0 \in (a, b) \colon f(x) = \sum_{k = 0}^n \frac{f^{(k)}(x_0)}{k!} (x - x_0)^k + \frac{1}{n!} \int_{x_0}^x f^{(n + 1)}(z) (x - z)^n\ dz.
    \]
  \end{thm}
  
  \begin{prf}
    $n = 0$: FTC: $f'$ exists and $f'$ is RI by assumption.
    \[
      f(x) = f(x_0) + \int_{x_0}^x f'(z)\ dz.
    \]
    
    $n \Rightarrow n + 1$: Assume $f^{(n + 2)}$ exists and is RI.
    Then $f^{(n + 1)}$ is continuous and therefore RI.
    Then 
    \begin{align*}
      \frac{1}{n!} \int f^{(n + 1)}(z) (x - z)^n\ dz &= \frac{1}{n!} \int f^{(n + 1)}(z) \frac{d}{dz} \left ( -\frac{(x - z)^{n + 1}}{n + 1} \right )\ dz \\ 
      &\overset{IBP}{=\joinrel=\joinrel=} \frac{1}{n!} f^{(n + 1)}(z) \left ( -\frac{(x - z)^{n + 1}}{n + 1} \right ) \Bigg |_{x_0}^x - \int_{x_0}^x f^{(n + 2)}(z) \left ( -\frac{(x - z)^{n + 1}}{n + 1} \right )\ dz \\ 
      &= \frac{f^{(n + 1)}(x_0)}{(n + 1)!} (x - x_0)^n + \int_{x_0}^x \frac{f^{(n + 2)}(z)}{(n + 1)!} (x - z)^{n + 1}\ dz.
    \end{align*}
    Then 
    \[
      f(x) - P_n(x) \overset{(n)}{=\joinrel=} \text{LHS} = P_{n + 1}(x) - P_n(x) + \int_{x_0}^x \frac{f^{(n + 2)}(z)}{n + 1} (x - z)^{n + 1}.
    \]
  \end{prf}
  
  \newpage
  
  {\boldmath \bfseries \underline{Stieljes integral}}
  
  {\boldmath \bfseries Idea:} Measure length of intervals using other functions than just $g(x) = x$.
  
  \begin{defn}
    Let $\Pi = (\left \{ t_i \right \}_{i = 0}^n, \left \{ t^*_i \right \}_{i = 1}^n)$ be a marked partition of $[a, b]$.
    For $f, g \colon [a, b] \to \mathbb R$, 
    \[
      S(f, dg, \Pi) := \sum_{i = 1}^n f(t^*_i) [g(t_i) - g(t_{i - 1})]
    \]
    is the {\boldmath \bfseries Riemann-Stieljes sum} of $f$ with respect to $g$.
  \end{defn}
  
  \begin{defn}
    Let $f, g \colon [a, b] \to \mathbb R$.
    We say that {\boldmath \bfseries ``$f$ is Stieljes integrable with respect to $g$ on $[a, b]$''} if 
    \[
      \exists \, L \in \mathbb R \, \forall \, \varepsilon > 0 \, \exists \, \delta > 0 \, \forall \, \Pi \text{ marked partition of $[a, b]$}: ||\Pi|| < \delta \Rightarrow \left | S(f, dg, \Pi) - L \right | < \varepsilon
    \]
    or in short, 
    \[
      \lim_{||\Pi|| \to 0} S(f, dg, \Pi) \text{ exists.}
    \]
  \end{defn}
  
  \begin{note}
    \begin{itemize}
      \item Such an $L$ is unique (if it exists) and so we denote it $\int_a^f f(x)\ dg(x) = \int_a^b f\ dg$.

      \item For $g(x) = x$, we get the Riemann integral.
        
      \item If ``length'' of $[t, s]$ is given by $g(s) - g(t)$, then $\int f\ dg$ corresponds to ``area'' with lengths in $\mathbb R$ measured using $g$.
        
      \item In probability: $g = \text{cumulative distribution function of a random variable $x$}$ ($g(t) := P(x \leq t)$) then 
      \[
        \int f(x)\ dg(x) = E(f(X)) = \text{expectation of $f(X)$}.
      \]
      
      \item In economics: $f(t) = \text{price of stock at time $t$}$, $g(t) = \text{current holding of the stock}$ then 
      \[
        \int_a^b f\ dg = \text{total money earned in time interval $[a, b]$.}
      \]
      This shows $g$ may not be monotone.
    \end{itemize}
  \end{note}
  
  \newpage
  
  \section{4.27 Wednesday Week 5}
  
  {\boldmath \bfseries \underline{Stieljes integral}}

  {\boldmath \bfseries Last time:} 
  \[
    S(f, dg, \Pi) = \sum_{i = 1}^n f(t^*_i) (g(t_i) - g(t_{i - 1}))
  \]
  \[
    \int_a^b f\ dg := \lim_{||\Pi|| \to 0} S(f, dg, \Pi) \text{ wherever it exists.}
  \]
  
  We call this the Stieljes integral in the Riemann sense.
  
  Notation: $RS(g, [a, b]) := \left \{ f \colon [a, b] \to \mathbb R : \int_a^b f\ dg \text{ exists} \right \}$ 
  
  \begin{lem}[Linearity]
    Let $h \colon [a, b] \to \mathbb R$ be given.
    Then 
    \[
      \forall \, f, g \in RS(h, [a, b]) \, \forall \, \alpha, \beta \in \mathbb R : \alpha f + \beta g \in RS(h, [a, b]) \land \int_a^b (\alpha f + \beta g)\ dh = \alpha \int_a^b f\ dh + \beta \int_a^b g\ dh.
    \]
  \end{lem}
  
  \begin{lem}[Additivity]
    Let $g \colon [a, b] \to \mathbb R$ be given.
    Then 
    \[
      \forall \, f \in RS(g, [a, b]) \, \forall \, c \in (a, b) : f \in RS(g, [a, b]) \land f \in RS(g, [c, b]) \land \int_a^b f\ dg = \int_a^c f\ dg + \int_c^b f\ dg.
    \]
  \end{lem}
  
  \begin{lem}
    Let $f \in RS(g, [a, b])$.
    Then 
    \[
      \left \{ x \in [a, b] : \text{$f$ discontinuous at $x$} \right \} \cap \left \{ x \in [a, b] : \text{$g$ discontinuous at $x$} \right \} = \varnothing.
    \]

    \begin{note}
      $f \in RS(g, [a, b])$ need not be bounded on intervals where $g$ is constant.
    \end{note}
  \end{lem}
  
  \begin{defn}
    We say $f$ is {\boldmath \bfseries generalized Stieljes integrable} with respect to $g$ if 
    \begin{align*}
      \exists \, L \in \mathbb R \, \forall \, \varepsilon > 0 \, \exists \, \delta > 0 \, \exists \, \Pi_\varepsilon \text{ unmarked partition} \, &\forall \, \Pi \text{ marked partition} : \\ 
      &||\Pi|| < \delta \land \Pi_\varepsilon \subseteq \Pi \Rightarrow \left | S(f, dg, \Pi) - L \right | < \varepsilon.
    \end{align*}
  \end{defn}
  
  {\boldmath \bfseries \underline{Criteria for Stieljes integrability}}
  
  \begin{thm}[Reduction to Riemann integral]
    Let $f, g \colon [a, b] \to \mathbb R$ be such that 
    \begin{enumerate}
      \item $f$ is Riemann integrable and 

      \item $g$ is cnotinuous on $[a, b]$, differentiable on $(a, b)$ with $g'$ Riemann integrable.
    \end{enumerate}
    Then 
    \[
      f \in RS(g, [a, b]) \quad \land \quad \int_a^b f\ dg = \int_a^b f(x) g'(x)\ dx.
    \]
  \end{thm}
  
  \begin{prf}
    Let $\Pi = (\left \{ t_i \right \}_{i = 0}^n, \left \{ t^*_i \right \}_{i = 1}^n)$ be a marked partition of $[a, b]$.
    For each $i = 1, \dots, n$, let $\tilde t_i$ be a point such that $g(t_i) - g(t_{i - 1}) = g'(\tilde t_i) (t_i - t_{i - 1})$ given by the mean value theorem.
    Let $\tilde \Pi = (\left \{ t_i \right \}_{i = 0}^n, \left \{ \tilde t_i \right \}_{i = 1}^n)$.
    Then 
    \begin{align*}
      S(f, dg, \Pi) - R(f g', \Pi') &= \sum_{i = 1}^n f(t^*_i) (g(t_i) - g(t_{i - 1})) - \sum_{i = 1}^n f(\tilde t_i) g(\tilde t_i) (t_i - t_{i - 1}) \\ 
      &= \sum_{i = 1}^n [f(t^*_i) - f(\tilde t_i)] g'(\tilde t_i) (t_i - t_{i - 1}).
    \end{align*}
    
    Note that 
    \[
      \left | \text{RHS} \right | \leq ||g'|| \sum_{i = 1}^n \operatorname{osc}(f, [t_{i - 1}, t_i]) (t_i - t_{i - 1}) \overset{\text{$f g'$ RI}}{\underset{||\Pi|| \to 0}{\longrightarrow}} 0. 
    \]
    
    Hence 
    \[
      \lim_{||\Pi|| \to 0} S(f, g, \Pi) = \lim_{||\Pi|| \to 0} R(f g', \Pi) = \int_a^b f g'\ dx.
    \]
  \end{prf}
  
  \begin{thm}[BV condition]
    Let $f, g \in [a, b] \to \mathbb R$ be such that 
    \begin{enumerate}
      \item $f$ is continuous and 
      
      \item $g$ is of bounded variation ($V(g, [a, b]) < \infty$).
    \end{enumerate}
    
    Then $f \in RS(g, [a, b])$ and 
    \[
      \left | \int_a^b f\ dg \right | \leq ||f|| V(g, [a, b]).
    \]
  \end{thm}
  
  \newpage
  
  \begin{prf}
    Let $\Pi = \left \{ t_i \right \}_{i = 0}^n$, $\tilde \Pi = \left \{ s_i \right \}_{j = 0}^m$ be unparked partitions of $[a, b]$.
    Assume $\Pi \subseteq \tilde \Pi$ and the set $J_i = \left \{ j = 1, \dots, m : [s_{j - 1}, s_j] \subseteq [t_{i - 1}, t_i] \right \}$.
    Now choose any marked points $t^*_i \in [t_{i - 1}, t_i]$ and $s^*_j \in [s_{j - 1}, s_j]$.
    Then 
    \begin{align*}
      S(f, dg, \Pi) - S(f, dg, \tilde \Pi) &= \sum_{i = 1}^n f(t^*_i) (g(t_i) - g(t_{i - 1})) - \sum_{j = 1}^m f(s^*_i) [g(s_j) - g(s_{j - 1})] \\ 
      &= \sum_{i = 1}^n \sum_{j \in J_i} [f(t^*_i) - f(s^*_j)] [g(s_j) - g(s_{j - 1})] \\ 
      &\leq \sum_{i = 1}^n \sum_{j \in J_i} \operatorname{osc}(f, [t_{i - 1}, t_i]) \left | g(s_j) - g(s_{j - 1}) \right | \\ 
      &\leq \sum_{i = 1}^n \operatorname{osc}(f, [t_{i - 1}, t_i]) V(g, [t_{i - 1}, t_i]).
    \end{align*}
    
    If $f$ is continuous then $f$ is uniformly continuous.
    Then 
    \[
      \forall \, \varepsilon > 0 \, \exists \, \delta > 0: ||\Pi|| < \delta \Rightarrow \operatorname{osc}(f, [t_{i - 1}, t_i]) < \varepsilon.
    \]
    Then $\left | \text{RHS} \right | \leq \varepsilon V(g, [a, b])$.
    Then for any marked partitions $\Pi, \Pi'$ of $[a, b]$ we have 
    \[
      ||\Pi||, ||\Pi'|| < \delta \Rightarrow \left | S(f, dg, \Pi) - S(f, dg, \Pi') \right | \leq 2 \varepsilon V(g, [a, b]).
    \]
  \end{prf}
  
  \begin{thm}[Loéve-Young condition, 1936]
    Let $f, g \colon [a, b] \to \mathbb R$ be such that 
    \[
      \exists \, \alpha, \beta \in (a, b]: \text{$f$ is $\alpha$-Hölder} \land \text{$g$ is $\beta$-Hölder} \land \alpha + \beta > 1.
    \]
    Then $f \in RS(g, [a, b])$.
    
    \begin{note}
      $f$ is $\alpha$-Hölder if 
      \[
        \exists \, C > 0 \, \forall \, x, y \in [a, b]: \left | f(x) - f(y) \right | \leq C \left | x - y \right |^\alpha.
      \]
    \end{note}
  \end{thm}
  
  \section{4.29 Friday Week 5}

  {\boldmath \bfseries \underline{Wrapping up Stieljes integral}}

  \begin{rmk}
    \begin{itemize}
      \item Stieljes integral includes sums: 
      \[
        F(x) = \sum_{i = 1}^n 1_{[x_i, \infty)} \text{ where } a < x_1 < x_2 < \cdots < x_n \leq b.
      \]
      Then for $g$ continuous: 
      \[
        \int_a^b g\ dF = \sum_{i = 1}^n g(x_i).
      \]
      We can combine these with the \textit{continuous part}: 
      \[
        F(x) = \sum_{i = 1}^n 1_{[x_i, \infty)}(x) + \int_a^x f(x)\ dt \Rightarrow \int_a^bb g\ dF = \sum_{i = 1}^r g(x_i) + \int_a^b g(t) f(t)\ dt.
      \]
      
      \item Standard facts apply: 

      \begin{lem}[Integration by parts]
        If $f \in RS(g, [a, b])$ and $g \in RS(f, [a, b])$ then 
        \[
          \int_a^b f\ dg + \int_a^b g\ df = f g \Big |_a^b = f(b) g(b) - f(a) g(a).
        \]
      \end{lem}
      
      \begin{lem}[Substitution]
        If $g \in RS(h, [a, b])$ and $G(x) := \int_a^x g\ dh$ then 
        \[
          f \in RS(G, [a, b]) \Leftrightarrow f g \in RS(h, [a, b])
        \]
        and if (both) true then 
        \[
          \int_a^b f\ dG = \int_a^b f g\ dh.
        \]
      \end{lem}
      
      \item The definition is unchanged if $f$ and $g$ are $\mathbb C$-valued.
      This allows us to define {\boldmath \bfseries curve integrals} 
      \[
        \int_\gamma f(x)\ dz := \int_0^1 f(\gamma(t))\ d\gamma(t)
      \]
      where $\gamma \colon [0, 1] \to \mathbb C$ continuous.

      This is independent of the parametrization.
      
      \item We can even generalize this to one of $f$ or $g$ being vector-valued and the other being scalar-valued.
        
      \item The length of a curve $\gamma \colon [a, b] \to X$ where $(X, \rho)$ is a metric space is given by 
      \[
        \operatorname{length}(\gamma) = \sup_{n \geq 1} \sup_{0 = t_0 < \cdots < t_n = 1} \sum_{i = 1}^n \rho(\gamma(t_{i - 1}), \gamma(t_i)).
      \]
      
      A curve is {\boldmath \bfseries rectifiable} if the length is finite.
      
      If $X = \mathbb R^n$ or some other normed space then 
      \[
        \rho(\gamma(t_{i - 1}), \gamma(t_i)) = || \gamma(t_i) - \gamma(t_{i - 1}) ||.
      \]
      This allows us to think of $\operatorname{length}(\gamma)$ as 
      \[
        \int_0^1 1\ d||\gamma||.
      \]
      
      If $\gamma$ is differentiable then 
      \[
        \gamma(t_i) - \gamma(t_{i - 1}) \approx \gamma'(t_{i - 1}) (t_i - t_{i - 1}).
      \]
      Then 
      \[
        \operatorname{length}(\gamma) = \int_0^1 || \gamma' ||(t)\ dt.
      \]
    \end{itemize}
  \end{rmk}
  
  {\boldmath \bfseries \underline{Extensions of Riemann-Stieljes theory}}
  
  {\boldmath \bfseries Lebesgue integral:} The idea is that instead of partitioning the domain of a function, partition the range.
  This requires developing a theory of measure of rather complicated sets.
  
  \begin{note}
    $f$ is Lebesgue integrable $\Rightarrow$ $\left | f \right |$ is Lebesgue integrable.
  \end{note}
  
  This is because Lebesgue integral mimics Darboux's approach.

  FTC II does not hold.

  The fix is given by: 
  \begin{defn}
    $f \colon [a, b] \to \mathbb R$ is said to be {\boldmath \bfseries Henstock-Kurzweil integrable} if 
    \begin{align*}
      \exists \, L \in \mathbb R \, &\forall \, \varepsilon > 0 \, \exists \, \delta \colon [a, b] \to (0, \infty) \, \forall \, \Pi = (\left \{ t_i \right \}_{i = 0}^n, \left \{ t^*_i \right \}_{i = 1}^n): \\ 
      &\forall \, i = 1, \dots, n: \left | t_i - t_{i - 1} \right | < \delta(t^*_i) \Rightarrow \left | R(f, \Pi) - L \right | < \varepsilon
    \end{align*}
    where $\delta$ is called the {\boldmath \bfseries guage function}.
  \end{defn}
  
  For bounded $f \colon [a, b] \to \mathbb R$, 
  \[
    f \text{ HK-integrable} \Leftrightarrow f \text{ measurable} \land f \text{ Lebesgue integrable}.
  \]
  
  FTC holds: suppose $F \colon [a, b] \to \mathbb R$ is differentiable on $(a, b)$.
  Then $F'$ is HK-integrable and 
  \[
    F(b) - F(a) = \int_a^b F(x)\ dx.
  \]
  
  However note that this is restricted to the real line since it uses a partition.
  
  {\boldmath \bfseries \underline{Uniform convergence}}

  {\boldmath \bfseries Q:} Let $\left \{ a_{n, m} \right \}_{n, m \in \mathbb N}$ be real such that 
  \[
    \forall \, m \in \mathbb N: b_m := \lim_{n \to \infty} a_{m, n} \text{ exists}
  \]
  and 
  \[
    \forall \, n \in \mathbb N: c_n := \lim_{m \to \infty} a_{m, n} \text{ exists.}
  \]
  
  When is $\lim_{n \to \infty} c_n = \lim_{m \to \infty} b_m$?
  
  \newpage
  
  \begin{lem}
    Suppose 
    \[
      \forall \, m \in \mathbb N\, \exists \, b_m \in \mathbb R: \lim_{n \to \infty} \sup_{n \in \mathbb N} \left | a_{m, n} - b_m \right | = 0, 
    \]
    or that $\lim_{m \to \infty} a_{m, n}$ is {\boldmath \bfseries uniform} in $n$.
    Then 
    \[
      \forall \, n \in \mathbb N: c_n := \lim_{m \to \infty} a_{m, n} \text{ exists} \Rightarrow \lim_{m \to \infty} b_m \text{ and } \lim_{n \to \infty} c_n \text{ exist} \land \lim_{n \to \infty} c_n = \lim_{m \to \infty} b_m.
    \]
    This means 
    \[
      \lim_{n \to \infty} \lim_{m \to \infty} a_{m, n} = \lim_{m \to \infty} \lim_{n \to \infty} a_{m, n}.
    \]
  \end{lem}
\end{document}